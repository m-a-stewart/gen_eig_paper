\documentclass{siamltex}
\usepackage[bookmarks=true]{hyperref}
% \usepackage{refcheck}
% \usepackage[margin=1.25in]{geometry}
\usepackage{mathtools}
\usepackage{amsmath}
% \usepackage{amsthm}
\usepackage{amsfonts}\usepackage{bm}
\usepackage{amssymb}
\usepackage{algorithm}
\usepackage[noEnd=false,indLines=false]{algpseudocodex}
\usepackage{colortbl}\usepackage{multirow}\usepackage{url}
\usepackage{placeins}
\usepackage{hhline}
%\usepackage{colonequals}
\usepackage{tikz}
\renewcommand{\topfraction}{1}
\renewcommand{\bottomfraction}{1}
\renewcommand{\textfraction}{0}
\DeclareMathOperator*{\argmin}{arg\,min}
\def\R{ {\mathbb R}}\def\C{ {\mathbb C}}\def\N{ {\mathbb N}}
\def\P{{\mathbb P}}\def\Z{{\mathbb Z}}
\def\alt{\mathrel{\raisebox{-.75ex}{$\mathop{\sim}\limits^{\textstyle <}$}}}
\def\ctp{^{\rm H}}\def\dia{{\rm diag}}\def\fro{_{\rm F}}\def\trp{^{\rm T}}
\def\trace{\operatorname{tr}}
\def\adj{^{*}}
\def\inv{^{-1}}\def\itp{^{\rm -T}}\def\unv{{\bf e}}\def\exc{{\rm exc}}
\def\real{\mbox{\rm Re}}\def\imag{\mbox{\rm Imag}}
\def\qed{\rule{1.2ex}{1.2ex}}\def\conj#1{\overline{#1}}
\def\fl{\mbox{fl}}\def\op{\, \mbox{op}\,}\def\sign{\mbox{sign}}
\def\eqand{\qquad\mbox{and}\qquad}
\def\rank{{\rm rank}}
\def\range{{\cal R}}
\def\nullspace{{\cal N}}
\def\vec#1{{\boldsymbol #1}}
\def\T{{\rm T}}
\def\pT{{\prime\, \T}}
\def\H{{\rm H}}
\def\th#1{\tilde{\hat{#1}}}
\def\e#1{{\times}10^{#1}}
\def\diag{\mbox{diag}}
\def\mcal{\mathcal}

\newenvironment{mtx}[1]{\left[\begin{array}{#1}}{\end{array}\right]}
\newtheorem{example}{Example}

% Put paragraph indentation in enumerate environment.
\let\oldenumerate=\enumerate
\renewenvironment{enumerate}{\oldenumerate\parindent=1.5em}{\endlist}

\bibliographystyle{siam}
 \title{Error Bounds for the Shift and Invert Solution
  of the Generalized Eigenvalue Problem}
\author{Michael Stewart\thanks{Department of Mathematics and
    Statistics, Georgia State University, Atlanta GA 30303, {\tt
      mastewart@gsu.edu}}}
\pagestyle{myheadings}
\thispagestyle{plain}
\markboth{MICHAEL STEWART}{SHIFT AND INVERT SOLUTION
  OF THE GENERALIZED EIGENVALUE PROBLEM}

\begin{document}
\maketitle
\begin{abstract}
  Given matrices $A$ and $B$, this paper gives error bounds for the solution of
  the generalized eigenvalue problem using a shift and invert strategy.  The
  analysis identifies circumstances under which the solution can be expected to
  satisfy useful error bounds.  The conditions are notably less restrictive than
  requiring that a shifted be well conditioned.
\end{abstract}
\begin{keywords}
  generalized eigenvalues, eigenvalues, eigenvectors, error analysis
\end{keywords}
\begin{AMS}
  15A18, 15A22, 15A23, 15A42, 65F15
\end{AMS}

\section{Background}
\label{sec:background}

In what follows we assume that $A$ and $B$ are $n\times n$ nonzero complex
matrices that are not necessarily hermitian.  Our goal is to solve the
generalized eigenvalue problem
\begin{equation}
  \label{eq:gen_eig1}
  A\vec{v}_i = \lambda B \vec{v}_i, \qquad \vec{v}_i \neq \vec{0}, \qquad
  i=1,2,\ldots n.
\end{equation}
We also use the formulation
\begin{equation}
  \label{eq:gen_eig2}
  \beta_i A \vec{v}_i = \alpha_i B \vec{v}_i, \qquad \vec{v}_i\neq \vec{0}, \qquad
  i = 1,2,\ldots, n.
\end{equation}
where $\beta_i$ and $\alpha_i$ are not both zero and
$\lambda_i = \alpha_i/\beta_i$.  If $\beta_i = 0$, we let $\lambda_i = \infty$.
If $A$ and $B$ have a common null space, then any common null vector would
satisfy \eqref{eq:gen_eig1} or \eqref{eq:gen_eig2} for any choice of $\lambda$
or $\alpha$ and $\beta$.  We therefore require that the intersection of the
null spaces of $A$ and $B$ be trivial, that is that the pencil $A-\lambda B$ be
\textit{regular}.  The problem is fully defined by the pair $(A,B)$ and the
generalized eigenvalues by $(\alpha_i, \beta_i)$.  When the context is clear,
we will often simply refer to eigenvalues and eigenvectors instead of
generalized eigenvalues and eigenvectors.  We also refer to both $\lambda_i$
and the pair $(\alpha_i, \beta_i)$ as eigenvalues.

If $B$ is nonsingular, then \eqref{eq:gen_eig1} can be converted to the
ordinary eigenvalue problem by solving the ordinary eigenvalue problem
$B^{-1}A$, which in practice would be computed using some stable factorization
of $B$.  If $A$ is nonsingular, then each eigenvalue of $A^{-1}B$ is of the
form $1/\lambda_i$, where $\lambda_i$ satisfies (\ref{eq:gen_eig1}).  Either
formulation can be used to solve (\ref{eq:gen_eig1}).  However, if the matrix
to be inverted is ill conditioned then the computed $B^{-1}A$ (or $A^{-1}B$)
might have large errors.  In this case, the computed generalized eigenvalues
might be inaccurate, even when they are well conditioned \cite{govl:13,
  stew:01}.

To try to avoid problems with ill conditioning of $A$ when forming $A^{-1}B$,
we can choose a shift for which $A-\sigma B$ is nonsingular and instead form
\begin{equation*}
X = (A-\sigma B)^{-1} B.
\end{equation*}
If
\begin{equation}
  \label{eq:X_problem}
  X \vec{v}_i = \theta_i \vec{v}_i, \qquad \vec{v}_i \neq 0, 
  \qquad i=1,2,\ldots, n,
\end{equation}
then (\ref{eq:gen_eig2}) holds with
\begin{equation*}
  \alpha_i = 1+\sigma \theta_i, \eqand \beta_i = \theta_i.
\end{equation*}
We then have
\begin{equation*}
  \lambda_i = (1+\sigma \theta_i)/\theta_i = \frac{1}{\theta_i} + \sigma
\end{equation*}
in (\ref{eq:gen_eig1}).  Note that either $\alpha_i$ or $\beta_i$ must be
nonzero.  This paper focuses on the numerical properties of the eigenvalue
problem (\ref{eq:X_problem}) as a means for solving (\ref{eq:gen_eig2}).
However, the error bounds depend on the eigenvalues $\lambda_i$ of
(\ref{eq:gen_eig1}) and on properties of the matrix $B^{-1}A$.\footnote{If $B$
  is nonsingular.  In \S\ref{sec:shift} we go to some effort to formulate
  theorems that also apply when $B$ is singular.}

Unfortunately, it is not always possible to choose $\sigma$ so that
$A-\sigma B$ is well conditioned.  Even with the use of a shift, the explicit
conversion of the generalized eigenvalue problem to an ordinary eigenvalue
problem is not commonly recommended, except perhaps in the use of iterative
methods like the Arnoldi algorithm.  The $QZ$ algorithm \cite{most:73} instead
solves the generalized eigenvalue by computing orthogonal $Q$ and $Z$ for which
\begin{equation*}
  A = Q T_a Z^\H, \eqand B = QT_b Z^\H
\end{equation*}
where $T_a$ and $T_b$ are upper triangular.  If the $i$th diagonal elements of
$T_a$ and $T_b$ are $\alpha_i$ and $\beta_i$ respectively, then each
$(\alpha_i, \beta_i)$ is an eigenvalue of \eqref{eq:gen_eig2}.  The
decomposition is perfectly backward stable so that for $i=1,\ldots, n$ the
pairs $(\alpha_i, \beta_i)$ form the set of eigenvalues of a pair of matrices
close to $(A,B)$.  The generalized eigenvectors can be obtained from $Z$ in a
manner analogous to what is done with a Schur decomposition for the ordinary
eigenvalue problem.  The only disadvantage of this method for a general dense
problem is the computational cost of the $QZ$ algorithm relative to that of
solving an ordinary eigenvalue problem for $(A-\sigma B)^{-1}B$.  However,
given the possible instability associated with forming $(A-\sigma B)^{-1}B$,
the $QZ$ algorithm is the standard method for the dense nonhermitian
generalized eigenvalue problem.

In this paper, we revisit the use of inversion to convert the generalized
eigenvalue problem to an ordinary eigenvalue problem and show that, while not
backward stable, it can be more stable than consideration of the condition
number of $A-\sigma B$ might suggest.  Further, the instability is detectable
prior to computing an eigenvalue decomposition of $X$ and can be quantified in
backward error and residual bounds.  In particular, if we define
\begin{equation}
  \label{eq:K2_C_def}
  K_2(C, B) = \frac{\|C\|_2 \|C^{-1} B\|_2}{\|B\|_2}
\end{equation}
then we will see in the error bounds that
\begin{equation}
  \label{eq:K2_def}
  K_2(A-\sigma B, B) = \frac{\|A-\sigma B\|_2}{\|B\|_2} \|(A-\sigma B)^{-1} B\|_2
\end{equation}
is the key factor affecting the stability, or lack of stability, of the shift
and invert approach.  If we define a scaled shift and scaled eigenvalues
\begin{equation}
  \label{eq:scaled_eig_shift}
  \hat{\sigma} = \frac{\|B\|_2}{\|A\|_2} \sigma, \qquad
  \hat{\lambda}_i = \frac{\|B\|_2}{\|A\|_2} \lambda_i, \qquad
  i = 1, 2,\ldots, n,
\end{equation}
and
\begin{equation*}
  \hat{A} = \frac{A}{\|A\|_2}, \eqand
  \hat{B} = \frac{B}{\|B\|_2},
\end{equation*}
then the condition number of $A-\sigma B$ satisfies $\kappa_2(A-\sigma B) = \kappa_2(\hat{A} - \hat{\sigma} \hat{B})$.
For $K_2(A-\sigma B, B)$, we have the similar identity
\begin{equation*}
  K_2(A-\sigma B, B)
  = \|\hat{A} - \hat{\sigma} \hat{B}\|_2
  \|(\hat{A}-\hat{\sigma} \hat{B})^{-1} \hat{B}\|_2
  = K_2(\hat{A}-\hat{\sigma} \hat{B}, \hat{B}).
\end{equation*}
Thus $K_2(A- \sigma B, B)$ does not change with scaling of $A$ and $B$, so long as
the shift is scaled accordingly.

It is shown in \S\ref{sec:shift} that
\begin{equation}
  \label{eq:upper_lower_bounds}
  \frac{\|\hat{A}-\hat{\sigma} \hat{B}\|_2}{\min_i |\hat{\lambda_i} - \hat{\sigma}|}
  \leq K_2(A-\sigma B,B) 
  \leq \kappa_2(\hat{A}-\hat{\sigma} \hat{B}),
\end{equation}
where the minimum is over all scaled generalized eigenvalues $\hat{\lambda}_i$
of the pair $(A,B)$, assuming $1/|\hat{\lambda}_i - \hat{\sigma}| = 0$ if
$\lambda_i = \infty$.  The upper bound is tight when
$\|(A-\sigma B)^{-1} B\|_2 = \|(A-\sigma B)^{-1} \|_2 \|B\|_2$.  The lower
bound will be seen to be an equality if $(A,B)$ has a complete orthonormal set
generalized eigenvectors.  If $B$ is nonsingular, this is equivalent to
$B^{-1}A$ being normal.  In the case in which $(A,B)$ has orthonormal
eigenvectors, it is often not too difficult to select a shift for which
$K_2(A - \sigma B, B)$ is not large.  The magnitude of $K_2(A-\sigma B, B)$ has
a pseudospectral characterization that we present in \S\ref{sec:shift}.


The structure of the paper is as follows.  In \S\ref{sec:shift} we consider the
quantity $K_2(A-\sigma B, B)$ in greater detail and relate it to the location
of the shift in relation to the a minor generalization of the pseudospectrum of
$B^{-1}A$ that is defined when $B$ is singular.  In
\S\ref{sec:algor-error-analys} we give error bounds for an algorithm that uses
shift and invert to compute a generalized Schur decomposition of $(A,B)$.  In
\S\ref{sec:residual-bounds} we present bounds that relate residuals for the
ordinary eigenvalue problem (\ref{eq:X_problem}) to residuals for the
generalized eigenvalue problem (\ref{eq:gen_eig2}).  The magnitude of
$K_2(A-\sigma B, B)$ is the key measure of instability in both
\S\ref{sec:algor-error-analys} and \S\ref{sec:residual-bounds}.

\section{Choosing a Shift}
\label{sec:shift}

In this section, we consider the choice of shift and its impact on
$K_2(A-\sigma B, B)$.  A rough summary of where we will end up is that we can
expect to have useful error bounds when $\sigma$ is not in the
$\epsilon$-pseudospectrum of $B^{-1}A$ for small $\epsilon$.  However, we do
not want to assume that $B$ is nonsingular, so we slightly modify some standard
results on pseudospectra to accommodate the possibility that $B$ is singular.
We do assume that $B\neq 0$.

Before trying to precisely characterize the magnitude of $K_2(A-\sigma B, B)$,
we first revisit the bounds from \S\ref{sec:background}.
\begin{lemma}
  The quantity $K_2(A-\sigma B, B)$ satisfies (\ref{eq:upper_lower_bounds}).
\end{lemma}
\begin{proof}
  The upper bound is immediate from
  \begin{multline*}
    K_2(A-\sigma B, B)
    = \frac{\|A-\sigma B\|_2}{\|B\|_2} \|(A-\sigma B)^{-1} B\|_2
    \leq \frac{\|A-\sigma B\|_2}{\|B\|_2} \|(A-\sigma B)^{-1}\|_2 \|B\|_2 \\
    = \|A-\sigma B\|_2 \|(A-\sigma B)^{-1}\|_2
    = \|\hat{A}-\hat{\sigma} \hat{B}\|_2 \|(\hat{A}-\hat{\sigma} \hat{B})^{-1}\|_2.
  \end{multline*}
  For the lower bound, we let $A = Q T_a Z^\H$ and $B = Q T_b Z^\H$ be a
  generalized Schur decomposition of $A$ and $B$, where the diagonal elements
  of $T_a$ are $\alpha_i$ and the diagonal elements of $T_b$ are $\beta_i$.  Then
  \begin{equation*}
    \|(A-\sigma B)^{-1} B\|_2 = \|T\|_2, \qquad T = (T_a - \sigma T_b)^{-1} T_b.
  \end{equation*}
  We then have
  \begin{equation*}
    \|(A-\sigma B)^{-1} B\|_2 
    \geq \rho(T) 
    = \max_i \left| \frac{\beta_i}{\alpha_i - \sigma \beta_i} \right|
    = \max_i \frac{1}{|\lambda_i - \sigma|},
    = \frac{1}{\min_i |\lambda_i - \sigma|},
  \end{equation*}
  where $\rho(T)$ is the spectral radius of $T$.  It should be understood that
  if $\beta_i=0$, then $\lambda_i = \infty$ and $1/|\lambda_i -\sigma|=0$.  With scaling, we have
  \begin{equation*}
    K_2(A-\sigma B, B)
    \geq \frac{\|A-\sigma B\|_2}{\|A\|_2} 
     \frac{1}{\min_i \|B\|_2 |\lambda_i - \sigma|/\|A\|_2}
     = \|\hat{A} - \hat{\sigma} \hat{B}\|_2
     \frac{1}{\min_i |\hat{\lambda}_i - \hat{\sigma}|}.
  \end{equation*}
\end{proof}

To interpret the lower bound, we define
\begin{equation}
  \label{eq:gamma_def}
  \gamma = \frac{\|A-\sigma B\|_2}{\|A\|_2}
  = \|\hat{A}-\hat{\sigma} \hat{B}\|_2 \leq 1+|\hat{\sigma}|
\end{equation}
in the lower bound is large only if the scaled shift $\hat{\sigma}$ is large.
This suggests that to moderate $K_2(A-\sigma B, B)$, we might need to limit the
magnitude of the shift.  However, for large shift we can note that
\begin{equation}
  \label{eq:upper_bound_for_lower_bound}
  \frac{\|\hat{A}-\hat{\sigma} \hat{B}\|_2}{\min_i |\hat{\lambda_i} - \hat{\sigma}|}
  \leq \frac{1+|\hat{\sigma}|}{\min_i |\hat{\lambda_i} - \hat{\sigma}|}
  = \frac{1+ 1/|\hat{\sigma}|}{\min_i |1 - \hat{\lambda_i}/\hat{\sigma}|}
  = \frac{1+ 1/|\hat{\sigma}|}{\min_i |1 - \lambda_i/\sigma|}
\end{equation}
so that for larger scaled shifts, the magnitude of the lower bound is
constrained by the relative distance of $\sigma$ from a generalized eigenvalue.
For small scaled shifts, the absolute distance is more important.

If $B$ is nonsingular, then the quantity $\|(A-\sigma B)^{-1} B\|_2$ used in
the definition of $K_2(A-\sigma B, B)$ is just the resolvent norm
$\|(B^{-1}A - \sigma I)^{-1}\|_2$.  This establishes a connection to the
pseudospectrum of $B^{-1}A$, \cite{trem:05}.  We now proceed to extend this
connection to the case in which $B$ might be singular.
\begin{lemma}
  \label{lm:pseudo-E}
  For $n\times n$ matrices $C$ and $B\neq 0$, let
  \begin{equation}
    \label{eq:E_def}
    \mathcal{E}(C,B) = 
    \left\{ E \mid \mbox{$C - B E$ is singular} \right\}.
  \end{equation}
  The set $\mcal{E}(C, B)$ is nonempty, closed, and contains $E_0$ for which
  \begin{equation*}
    \|E_0\|_2 = \min_{E\in\mcal{S}(C,B)} \|E\|_2.
  \end{equation*}
\end{lemma}
\begin{proof}
  To see that $\mcal{E}(C,B)$ is nonempty, we note that if $C$ is singular,
  then $E=0$ is in $\mcal{E}(C,B)$.  If $C$ is nonsingular, we choose any
  $\vec{y} \neq \vec{0}$ satisfying $B\vec{y} \neq \vec{0}$ and define
  $\vec{x}$ by $C\vec{x} = B\vec{y}$.  This guarantees $\vec{x}\neq \vec{0}$.
  Let $E = \vec{y}\vec{x}^\H / \|\vec{x}\|_2^2$ so that
  \begin{equation*}
    C - BE = C - \frac{1}{\|\vec{x}\|_2^2} C\vec{x} \vec{x}^\H =
    C \left( I - \frac{1}{\|\vec{x}\|_2^2} \vec{x} \vec{x}^\H\right),
  \end{equation*}
  which has $\vec{x}$ as a null vector.

  It is easily seen that the complement
  $\mathcal{E}(C,B) = \left\{ E \mid \mbox{$C - B E$ is nonsingular} \right\}$
  is open, so that $\mcal{E}(C,B)$ is closed.  Given $E_1 \in \mcal{E}(C,B)$,
  the set 
  \begin{equation*}
    \mcal{F}(C,B) = \left\{ E \mid \mbox{$C - B E$ is singular and $\|E\| \leq \|E_1\|$}
    \right\}
  \end{equation*}
  is similarly seen to be closed, so it is compact and contains $E_0$ of
  minimal norm, which is then also of minimal norm of all $E\in \mcal{E}(C,B)$.
\end{proof}

We are interested in the case $C = A-\sigma B$.  Assuming $A$ and $B$ are given
and $\sigma \in \C$, we can define
\begin{equation}
  \label{eq:delta_def}
  \delta(\sigma) 
  = \min_{A-\sigma B - BE\,\text{is singular}} \|E\|_2
  = \min_{E \in \mcal{E}(A-\sigma B, B)} \|E\|_2
\end{equation}
The quantity $\delta(\sigma)$ measures the smallest $\|E\|_2$ for which
$\sigma$ is an eigenvalue of $(A-BE, B)$.

For $\epsilon \geq 0$ we can define
\begin{equation}
  \label{eq:pseudo_def}
  \Lambda_\epsilon = \left\{ \sigma \in \C \mid \delta(\sigma) < \epsilon
  \right\}.
\end{equation}
When $B$ happens to be nonsingular, this is the $\epsilon$-pseudospectrum of
$B^{-1}A$.  The set of finite generalized eigenvalues of $(A,B)$ is given by
$\Lambda_0$.  For $\epsilon \geq 0$, $\Lambda_\epsilon$ contains the finite
generalized eigenvalues of $(A,B)$.

If $B$ is nonsingular and $\sigma$ is not an eigenvalue of $B^{-1}A$, then it
is well known that $\sigma$ is in the $\epsilon$-pseudospectrum of $B^{-1}A$ if
and only if $\|(B^{-1}A - \sigma I)^{-1}\|_2 > 1/\epsilon$.  This is
equivalent to $\delta(\sigma) = 1/\|(B^{-1}A - \sigma I)^{-1}\|_2$.
The following lemma reformulates this result to avoid assuming
invertibility of $B$.
\begin{lemma}
  \label{lm:X_norm_bound}
  With $\delta(\sigma)$ defined as in \eqref{eq:delta_def}, we have
  \begin{equation*}
     \delta(\sigma) 
     = 
     \left\{
       \begin{array}{ll}
         \frac{1}{\|(A-\sigma B)^{-1} B\|_2}, & \sigma \notin \Lambda(0) \\
         0, & \mbox{otherwise} \\
       \end{array}
     \right..
  \end{equation*}
\end{lemma}
\begin{proof}
  If $A-\sigma B$ is singular, it is clear from \eqref{eq:pseudo_def}
  and \eqref{eq:delta_def} that $\delta(\sigma)=0$.  If $A-\sigma B$
  is nonsingular we consider the generalized singular value
  decomposition
  \begin{equation*}
    A - \sigma B = S I U^\H,\eqand
    B = S D V^\H
  \end{equation*}
  where $S=(A-\sigma B)U$ is invertible,
  $(A-\sigma B)^{-1} B = U D V^\H$, $U$ and $V$ are unitary, and
  $D = \diag(d_1, \ldots, d_n)$ with $d_1 > 0$ and $d_k\geq 0$
  nonincreasing for $k=1,2,\ldots, n$.   Clearly
  $A-\sigma B - BE = S (I - D V^\H E U) U^\H$ is
  singular if and only if $(I - D V^\H E U)$ is singular which implies
  that $\|E\|_2 \|D\|_2 \geq 1$ and
  \begin{equation*}
    \delta(\sigma)
    \geq \frac{1}{d_1} =
    \frac{1}{\|(A-\sigma B)^{-1}B\|_2}.
  \end{equation*}
  Let  
  \begin{equation*}
    \hat{E} = \frac{1}{d_1}V \vec{e}_1 \vec{e}_1^\H U^\H.
  \end{equation*}
  Then
  \begin{equation*}
    A-\sigma B - B \hat{E} = S (I-\vec{e}_1 \vec{e}_1^\H) U^\H
  \end{equation*}
  is singular so that
  \begin{equation*}
    \delta(\sigma)
    \leq \|\hat{E}\|_2 
    = \frac{1}{d_1} = \frac{1}{\|(A-\sigma B)^{-1} B\|_2}.
  \end{equation*}
\end{proof}

If $B^{-1}A$ is normal then it is easily seen that
$\delta(\sigma) = \min_i |\lambda_i - \sigma|$, where the minimum is taken over
all generalized eigenvalues $\lambda_i$ of $(A,B)$.  This also generalizes to
the case in which $B$ is singular.
\begin{lemma}
  \label{lm:normal_eig_bound}
  Assume that $A$ and $B$ have no common null space and $B\neq 0$.  If the pair
  $(A,B)$ has an orthonormal basis of generalized eigenvectors $\vec{v}_i$,
  $i = 1, \ldots, n$, then
  \begin{equation}
    \label{eq:normal_delta}
    \delta(\sigma) = \min_i |\lambda_i - \sigma|.
  \end{equation}
\end{lemma}
\begin{proof}
  The claim follows immediately with $\delta(\sigma)=0$ if $\sigma$ is a
  generalized eigenvalue of $(A,B)$.  We assume that this is not the case so
  that $A-\sigma B$ is nonsingular.  Let $V$ be the unitary matrix with $i$th
  column $\vec{v}_i$.  For each generalized eigenvector we have
  $\beta_i A\vec{v}_i = \alpha_i B \vec{v}_i$, where regularity ensures that we
  do not have both $A\vec{v}_i=\vec{0}$ and $B\vec{v}_i= \vec{0}$.  It follows
  that the set $\{A\vec{v}_i, B\vec{v}_i\}$ spans a subspace of dimension one.
  Let $\vec{s}_i\neq \vec{0}$ be in this space and let $S$ be the matrix with
  $i$th column $\vec{s}_i$.  We then have
  \begin{equation*}
    AV = S D_a, \eqand
    BV = S D_b
  \end{equation*}
  for some diagonal matrices $D_a$ and $D_b$.  Thus
  \begin{equation*}
    (A-\sigma B) V = S (D_a - \sigma D_b).
  \end{equation*}
  Since $A-\sigma B$ is nonsingular and $V$ is unitary, $S$ must be
  nonsingular.  Let the diagonal elements of $D_a$ be $\hat{\alpha}_i$ and
  those of $D_b$ be $\hat{\beta}_i$ for $i=1,\ldots, n$.  We have
  \begin{align*}
    A - \sigma B - BE 
    & = S (D_a - \sigma D_b - D_b V^\H E V) V^\H \\
    & = S (D_a - \sigma D_b)\left( I - (D_a - \sigma D_b)^{-1} D_b V^\H E V \right) V^\H.
  \end{align*}
  This is singular only if $I - (D_a - \sigma D_b)^{-1} D_b V^\H E V$ is singular
  which implies
  \begin{equation*}
    \|(D_a - \sigma D_b)^{-1} D_b\|_2 \|V^\H E V\|_2 \geq \|(D_a - \sigma D_b)^{-1} D_b V^\H E V\|_2 \geq 1.
  \end{equation*}
  We have
  \begin{equation*}
    (D_a - \sigma D_b)^{-1} D_b = \diag\left(\frac{1}{(\lambda_1-\sigma)},
      \ldots, \frac{1}{(\lambda_n-\sigma)}\right)
  \end{equation*}
  so that
  \begin{equation}
    \label{eq:E_lower_bound}
    \|E\|_2 \geq\frac{1}{\|(D_a - \sigma D_b)^{-1} D_b\|_2} 
    = \frac{1}{\max_i 1/|\lambda_i - \sigma|} = \min_i |\lambda_i - \sigma|.
  \end{equation}
  The pair $(A,B)$ has a finite generalized eigenvalue for each $\hat{\beta}_i$
  satisfying $\hat{\beta}_i\neq 0$.  Since $B = S D_b V^\H\neq 0$, we must have
  at least one nonzero $\hat{\beta}_i$ so that $\min_i |\lambda_i - \sigma|$ is
  finite.  Without loss of generality, assume that the columns of $V$ and $S$
  are permuted so that $|\lambda_1 - \sigma| = \min_i |\lambda_i - \sigma|$ and
  $\hat{\beta}_1\neq 0$.  Then
  \begin{equation*}
    E = \frac{\hat{\alpha}_1 - \sigma \hat{\beta}_1}{\hat{\beta}_1} V \vec{e}_1 \vec{e}_1^\T V^\H =
    (\lambda_1 - \sigma) V \vec{e}_1 \vec{e}_1^\T V^\H
  \end{equation*}
  ensures that $ I - (D_a - \sigma D_b)^{-1} D_b V^\H E V$ is singular with
  $\vec{e}_1$ as a null vector.  Hence $A-\sigma B - BE$ is singular and, since
  it achieves the lower bound (\ref{eq:E_lower_bound}), this choice of $E$ also
  has minimal norm.
\end{proof}

We return to the quantity $K_2(A-\sigma B, B)$ defined in (\ref{eq:K2_def}).
From Lemma~\ref{lm:X_norm_bound} we know that
\begin{equation*}
  K_2(A-\sigma B, B) = \frac{\|A-\sigma B\|_2}{\|B\|_2} \frac{1}{\delta(\sigma)}.
\end{equation*}
The scaling in this expression is inconvenient.  To fix this, we note that
\begin{align*}
  K_2(A-\sigma B, B) 
  & = \frac{\|A-\sigma B\|_2}{\|B\|_2} \|(A-\sigma B)^{-1}B\|_2 \\
  & = \left( \frac{\|A-\sigma B\|_2}{\|B\|_2} \frac{\|B\|_2}{\|A\|_2}\right) 
    \left( \frac{\|A\|_2}{\|B\|_2}\|(A-\sigma B)^{-1}B\|_2  \right)\\
  & = \left\| \frac{A}{\|A\|_2} - \sigma \frac{\|B\|_2}{\|A\|_2}
    \frac{B}{\|B\|_2}
    \right\|_2 \left\|\left(\frac{A}{\|A\|_2}
    - \sigma \frac{\|B\|_2}{\|A\|_2} \frac{B}{\|B\|_2}\right)^{-1}
    \frac{B}{\|B\|_2}\right\|_2 \\
  & = \|\hat{A} - \hat{\sigma} \hat{B}\|_2 
    \|(\hat{A} - \hat{\sigma} \hat{B})^{-1} \hat{B}\|_2
\end{align*}
where $\hat{A} = A/\|A\|_2$, $\hat{B} = B/\|B\|_2$, and $\hat{\sigma}$ is the
scaled shift in (\ref{eq:scaled_eig_shift}).  This suggests a scaled
version of $\delta(\sigma)$, so we define
\begin{equation*}
  \hat{\delta}(\hat{\sigma}) 
  = \|(\hat{A} - \hat{\sigma} \hat{B})^{-1} \hat{B}\|_2
  = \frac{\|B\|_2}{\|A\|_2}\frac{1}{\|(A-\sigma B)^{-1}B\|_2}
  = \frac{\|B\|_2}{\|A\|_2} \delta(\sigma).
\end{equation*}
We then have
\begin{equation}
  \label{eq:K2_delta_ident}
  K_2(A-\sigma B, B) 
  = \|\hat{A} - \hat{\sigma} \hat{B}\|_2 \frac{1}{\hat{\delta}(\hat{\sigma})}
  \leq \frac{1+|\hat{\sigma}|}{\hat{\delta}(\hat{\sigma})}
  = \frac{1+1/|\hat{\sigma}|}{\hat{\delta}(\hat{\sigma})/|\hat{\sigma}|}
\end{equation}
where the final identity assumes that $\hat{\sigma} \neq 0$.  The two forms of
the upper bound in \eqref{eq:K2_delta_ident} are analogous to
\eqref{eq:upper_bound_for_lower_bound}.  For larger values of $\hat{\sigma}$,
$K_2(A-\sigma B, B)$ remains moderate if $\hat{\delta}(\hat{\sigma})$ is not
too small relative to $|\hat{\sigma}|$.

Finally, we can bound $K_2(A-\sigma B, B)$ in terms of the condition number of
the matrix of generalized eigenvectors.
\begin{lemma}
  \label{lm:bauer_fike_bound}
  Let $B\neq 0$ and $\sigma \notin \Lambda_0$.  Assume that the columns of $V$ give a
  complete set of eigenvectors for $(A,B)$.  Then
  \begin{equation*}
    K_2(A-\sigma B, B) 
    \leq \kappa_2(V) \frac{\|\hat{A} - \hat{\sigma} \hat{B}\|_2}
    {\min_i |\hat{\lambda}_i - \hat{\sigma}|}
    = \kappa_2(V) \frac{1+|\hat{\sigma}|}{\min_i |\hat{\lambda}_i - \hat{\sigma}|}
    = \kappa_2(V) \frac{1+1/|\hat{\sigma}|}{\min_i |1 - \hat{\lambda}_i / \hat{\sigma}|}.
  \end{equation*}
\end{lemma}
\begin{proof}
  The proof uses the same approach as the start of the proof of
  Lemma~\ref{lm:normal_eig_bound}.  Let
  $\beta_i A \vec{v}_i = \alpha_i B \vec{v_i}$ with $\alpha_i$ and $\beta_i$ no
  both zero.  As in Lemma~\ref{lm:normal_eig_bound} we can construct an $S$
  such that $A V = SD_a$ and $BV= SD_b$ where the diagonal elements of $D_a$
  are $\hat{\alpha}_i$ and those of $D_b$ are $\hat{\beta}_i$.  We then have
  \begin{multline*}
    \|(A-\sigma B)^{-1} B\|_2
    = \|V (D_a - \sigma D_b)^{-1} D_b V^{-1}\|_2
    \leq \kappa_2(V) \|(D_a - \sigma D_b)^{-1} D_b \|_2 \\
    = \kappa_2(V) \frac{1}{\min_i | \lambda_i - \sigma|}.
  \end{multline*}
  The bounds involving scaled matrices and quantities are given by the same
  manipulation we have used in \eqref{eq:upper_bound_for_lower_bound}.
\end{proof}

\section{Algorithm and Error Analysis}
\label{sec:algor-error-analys}

We now turn our attention to the computation of a generalized Schur
decomposition of $(A,B)$ from a schur decomposition of
$X = (A-\sigma B)^{-1} B$.  Suppose that $\sigma \notin \Lambda_0$
and that we have a Schur decomposition $X = ZTZ^\H$.  We then have
\begin{equation*}
  (A -\sigma B) Z T = B Z,\qquad\mbox{or}\qquad
  A Z T = B Z (I +\sigma T).
\end{equation*}
If we compute a $QR$ factorization $(A -\sigma B)Z = QT_1$, then $T_1$
is nonsingular and $Q^\H (A-\sigma B) Z = T_1$ so that
$Q^\H BZ = T_1 T$.  This leads to a generalized Schur decomposition
\begin{equation}
  \label{eq:Gen_schur_decomp}
  Q^\H A Z = T_1(I+\sigma T),\eqand Q^\H BZ = T_1 T.
\end{equation}
Since $T_1$ is nonsingular, the generalized eigenvalues of $(A,B)$ are the same
as those of $(I+\sigma T, T)$.  If the diagonal elements of $T$ are $\theta_i$
for $i=1,\ldots n$, then the eigenvalues of $(A,B)$ are
$(1+\sigma\theta_i, \theta_i)\neq (0,0)$.  The finite generalized eigenvalues
are $\lambda_i = (1+\sigma \theta_i)/\theta_i$ for each $\theta_i\neq 0$.  The
first $k$ columns of $Z$ give an orthonormal basis for a deflating subspace of
$(A,B)$.  This approach leads to Algorithm~\ref{alg:gen_schur}.

If needed, eigenvectors can be computed in a manner that is similar to
any other generalized Schur decomposition.  Specifically, if
$\vec{u}_i$ is an eigenvector for $(I+\sigma T, T)$ corresponding to
eigenvalue $(1+\sigma\theta_i, \theta_i)$, then
\begin{equation*}
  \theta_i A Z \vec{u}_i 
  = \theta_i QT_1(I+\sigma T) \vec{u}_i
  = (1+\sigma \theta_i) Q T_1 T \vec{u}_i
  = (1+\sigma \theta_i) B Z \vec{u}_i
\end{equation*}
so that $\vec{v}_i = Z \vec{u}_i$ is an eigenvector for $(A,B)$.  

\begin{algorithm}
\caption{Shift and Invert Schur Decomposition}
\label{alg:gen_schur}
\begin{algorithmic}
\Function{GenSchur}{$A, B, \sigma$}
\State $\breve{A} \gets A - \sigma B$
\State Solve: $\breve{A}X = B$
\State Factor: $X = Z T Z^{\H}$
\State $W \gets \breve{A}Z$
\State Factor: $W = Q T_1$
\State \Return $(T, T_1, Q, Z)$
\EndFunction
\end{algorithmic}
\end{algorithm}

The following lemma is useful in the derivation of backward error
bounds for Algorithm~\ref{alg:gen_schur}.
\begin{lemma}
  \label{lm:pseudo_inv}
  For a given shift $\sigma$ with $\hat{\sigma}$ defined by
  \eqref{eq:scaled_eig_shift} and arbitrary matrix $T$, let
  \begin{equation*}
    C
    =
    \begin{bmatrix}
      -\|A\|_2 T / \|B\|_2 \\ I + \sigma T
    \end{bmatrix}.
  \end{equation*}
  Then the pseudoinverse $C^\dagger$ of $C$ satisfies
  \begin{equation*}
    \|C^{\dagger}\|_2\leq (1+|\hat{\sigma}|)
  \end{equation*}.
\end{lemma}
\begin{proof}
  We seek a lower bound on the smallest singular value of $C$.  We consider two
  cases.  Given a vector $\vec{x}$ with $\|\vec{x}\|_2 = 1$, either
  \begin{equation*}
    \|T\vec{x}\|_2 \leq \frac{\|B\|_2}{(1+|\hat{\sigma}|)\|A\|_2},
    \qquad\mbox{or}\qquad
    \|T\vec{x}\|_2 > \frac{\|B\|_2}{(1+|\hat{\sigma}|)\|A\|_2}.
  \end{equation*}
  In the first case we have
  \begin{equation*}
    \|(I + \sigma T) \vec{x}\|_2
    \geq \|\vec{x}\|_2 - |\sigma| \|T\vec{x}\|
    \geq 1 - \frac{|\sigma| \|B\|_2}{(1+|\hat{\sigma}|)\|A\|_2}
    = 1 - \frac{|\hat{\sigma}|}{(1+|\hat{\sigma}|)}
    = \frac{1}{1+|\hat{\sigma}|}.
  \end{equation*}
  The second case immediately gives
  \begin{equation*}
    \frac{\|A\|_2\|T\vec{x}\|_2}{\|B\|_2}  > \frac{1}{1+|\hat{\sigma}|}.
  \end{equation*}
  It follows that $\|C\vec{x}\| \geq 1/(1+|\hat{\sigma}|)$ for arbitrary
  $\vec{x}$ with $\|\vec{x}\|_2=1$ so that
  $\sigma_n(C)\geq 1/(1+|\hat{\sigma}|)$ and
  $\|C^{\dagger}\|_2\leq (1+|\hat{\sigma}|)$.
\end{proof}

\begin{theorem}
  \label{th:generalized_schur}
  Suppose that $A-\sigma B$ is nonsingular, and that there exist
  $a_n$, $b_n$, $c_n$, $d_n$, $e_n$, $f_n$, and $g_n$ depending solely
  on $n$ for which
  \begin{equation}
    \label{eq:shift_error}
    \|\breve{A} - (A-\sigma B)\|_2 \leq u a_n (\|A-\sigma B\|_2 + |\sigma| \|B\|_2),
  \end{equation}
  \begin{equation}
    \label{eq:Xerror}
    \left\| \breve{A} X - B \right\|_2 
    \leq u b_n \|\breve{A}\|_2\|X\|_2,
  \end{equation}
  \begin{equation}
    \label{eq:schur_error}
    \| X - \tilde{Z} T \tilde{Z}^\H \|_2 \leq u c_n \|X\|_2,\qquad
    \|Z - \tilde{Z}\|_2 \leq u d_n,
  \end{equation}
  \begin{equation}
    \label{eq:AZerror}
    \|\breve{A} Z - W\|_2 \leq u e_n \|\breve{A}\|_2,
  \end{equation}
  \begin{equation}
    \label{eq:QRerror}
    \|\tilde{Q}T_1 - W\|_2 \leq u f_n \|W\|_2, \eqand
    \|\tilde{Q} - Q\|_2 \leq ug_n ,
  \end{equation}
  where $\tilde{Q}$ and $\tilde{Z}$ are unitary, $T_1$ and $T$ are
  upper triangular, and $T$ is nonsingular.  Then there exist $E$ and
  $F$ satisfying
  \begin{equation*}
    \tilde{Q}^\H (A+E) \tilde{Z} = T_1(I+\sigma T),\eqand \tilde{Q}^\H (B+F) \tilde{Z}
    = T_1 T.
  \end{equation*}
  with
  \begin{alignat*}{3}
    \frac{\|E\|_2}{\|A\|_2} & \leq
    u \Big(&& |\hat{\sigma}| (b_n + c_n + d_n + e_n + f_n) K_2(A-\sigma B, B) \\
    & && + (1+|\hat{\sigma}|)(2a_n + d_n+e_n+f_n)\Big) +O(u^2)    
  \end{alignat*}
  and
  \begin{equation*}
    \frac{\|F\|_2}{\|B\|_2} \leq
    u \left( b_n + c_n +d_n +e_n + f_n\right) K_2(A-\sigma B, B) + O(u^2).
  \end{equation*}
\end{theorem}
\begin{proof}
  Let $\breve{A} X - B = G_1$, $X - \tilde{Z}T \tilde{Z}^\H = G_2$,
  $Z - \tilde{Z} = G_3$, $\breve{A}Z - W = G_4$, and
  $\tilde{Q}T_1 - W = G_5$.  The first two of these identities give
  \begin{equation*}
    \breve{A} \tilde{Z} T = B\tilde{Z} + G_1 \tilde{Z} - \breve{A}G_2 \tilde{Z}.
  \end{equation*}
  Continuing with various substitutions, we have
  \begin{equation*}
    \breve{A} Z T = B\tilde{Z} + G_1 \tilde{Z} - \breve{A}G_2 \tilde{Z} + \breve{A} G_3 T,
  \end{equation*}
  \begin{equation*}
    W T = B\tilde{Z} + G_1 \tilde{Z} - \breve{A}G_2 \tilde{Z} + \breve{A} G_3 T - G_4 T,
  \end{equation*}
  and
  \begin{equation*}
    \tilde{Q} T_1 T = B\tilde{Z} + G_1 \tilde{Z} - \breve{A}G_2 \tilde{Z} + \breve{A} G_3 T - G_4 T + G_5 T.
  \end{equation*}
  This implies that
  \begin{equation*}
    B + F = \tilde{Q} T_1 T \tilde{Z}^\H
  \end{equation*}
  where
  \begin{align*}
    \|F\|_2
    & \leq \|G_1\|_2 + \|\breve{A}\|_2 \|G_2\|_2 + \|\breve{A}\|_2 \|G_3\|_2 \|T\|_2 +
      \|G_4\|_2 \|T\|_2 + \|G_5\|_2 \|T\|_2 \\
    & = \begin{aligned}[t] & \|G_1\|_2 + \|\breve{A}\|_2 \|G_2\|_2 
                           + \|\breve{A}\|_2 \|G_3\|_2 \|X\|_2 +
                           \|G_4\|_2 \|X\|_2 + \|G_5\|_2 \|X\|_2   \\
                           & + O(u^2) \\
    \end{aligned} \\
    & \leq \begin{aligned}[t] u ( & b_n \|\breve{A}\|_2 \|X\|_2 
                                    + c_n \|\breve{A}\|_2 \|X\|_2 
                                    + d_n \|\breve{A}\|_2 \|X\|_2\\
                                  & \!+ e_n \|\breve{A}\|_2 \|X\|_2 
                                    + f_n \|W\|_2 \|X\|_2) +O(u^2)
      \end{aligned} \\
    & = u \left( b_n + c_n +d_n +e_n + f_n\right) \|A-\sigma B\|_2 \|X\|_2 + O(u^2) \\
    & = u \left( b_n + c_n +d_n +e_n + f_n\right) K_2(A-\sigma B, B) \|B\|_2 + O(u^2).
  \end{align*}
  
  Let $\breve{A} = A-\sigma B + G_0$.  Considering errors on $A$, we
  start with $W - G_5 = \tilde{Q}T_1$ and continue with substitutions
  \begin{equation*}
    \breve{A}Z - G_4 - G_5 = \tilde{Q}T_1,
  \end{equation*}
  \begin{equation*}
    (A-\sigma B + G_0)Z - G_4 - G_5 = \tilde{Q}T_1,
  \end{equation*}
  and
  \begin{equation*}
    (A-\sigma B + G_0)\tilde{Z} - G_4 + (A-\sigma B+G_0) G_3 - G_5  = \tilde{Q}T_1.
  \end{equation*}
  This implies that
  \begin{multline*}
    \tilde{Q}^\H \left(A + \sigma F + G_0 + \tilde{Q} \left[(A-\sigma B+G_0) G_3- G_4 -G_5\right] \tilde{Z}^\H\right)
    \tilde{Z} \\
    = T_1 + \sigma \tilde{Q}^\H(B+F) \tilde{Z} = T_1 (I+\sigma T),
  \end{multline*}
  or
  \begin{equation*}
    \tilde{Q}^\H (A+E) \tilde{Z} = T_1(I+\sigma T)
  \end{equation*}
  where
  \begin{align*}
    \|E\|_2 
    & \leq |\sigma| \|F\|_2 + \|G_0\|_2 + \|A-\sigma B\|_2 \|G_3\|_2 + \|G_4\|_2 + \|G_5\|_2 + O(u^2) \\
    & \leq 
      \begin{aligned}[t]
        & u |\sigma| (b_n + c_n + d_n + e_n + f_n) K_2(A-\sigma B, B) \|B\|_2 
          + ua_n |\sigma| \|B\|_2 \\
        & \!+ u(a_n + d_n+e_n+f_n) \|A-\sigma B\|_2 +O(u^2)\\
      \end{aligned} \\
    & \leq 
      \begin{aligned}[t] 
        & u |\hat{\sigma}| (b_n + c_n + d_n + e_n + f_n) 
          K_2(A-\sigma B, B) \|A\|_2 + ua_n |\hat{\sigma}| \|A\|_2 \\
        & \! + u(a_n + d_n+e_n+f_n) (\|A\|_2 + |\hat{\sigma}| \|A\|_2) +O(u^2)\\
      \end{aligned} \\
    & \leq 
      \begin{aligned}[t] 
        u \Big(& |\hat{\sigma}| (b_n + c_n + d_n + e_n + f_n) K_2(A-\sigma B, B) \\
        & \! + (1+|\hat{\sigma}|)(2a_n + d_n+e_n+f_n)\Big) \|A\|_2 +O(u^2).\\
      \end{aligned} \\
  \end{align*}
\end{proof}

Algorithm~\ref{alg:gen_schur} computes a generalized Schur decomposition from
$B$ and $A-\sigma B$.  Recalling \eqref{eq:gamma_def}, we note that
$1/\gamma = \|A\|_2 / \|A-\sigma B\|_2$ is large only if there is extreme
cancellation in forming $A-\sigma B$.  Cancellation did not turn out to be a
concern in Theorem~\ref{th:generalized_schur}, but the quantity
$|\hat{\sigma}|/\gamma$ will appear in the residual bounds.  This quantity is
large only if $1/\gamma$ is large, regardless of the magnitude of
$|\hat{\sigma}|$.  To see this note that
\begin{equation*}
  \frac{|\hat{\sigma}|}{\gamma}
  = \frac{\hat{\sigma}}{\|\hat{A} - \hat{\sigma} \hat{B}\|_2}
  \leq \frac{|\hat{\sigma}|}{| \|\hat{A}\|_2 - |\hat{\sigma}| \|\hat{B}\|_2|}
  = \frac{|\hat{\sigma}|}{| 1 - |\hat{\sigma}| |}.
\end{equation*}
If $|\hat{\sigma}| \leq 2$, then $|\hat{\sigma}/\gamma| \leq 2/\gamma$.  If
$|\hat{\sigma}| > 2$, then the above inequality implies that
$|\hat{\sigma}| / \gamma \leq 2$.  Thus, with no assumption on $\hat{\sigma}$,
we have
\begin{equation*}
  \label{eq:sigma0_gamma_bound}
  \frac{|\hat{\sigma}|}{\gamma} \leq 2\max\left(\frac{1}{\gamma}, 1\right).
\end{equation*}


\begin{theorem}
  \label{th:residual_bounds}
  Assume that \eqref{eq:shift_error} and \eqref{eq:Xerror} hold.
  Suppose that a computed eigenvalue $\theta\neq 0$ and eigenvector
  $\vec{v}$ of $X$ satisfy
  \begin{equation*}
    X\vec{v} - \theta \vec{v} = \vec{r},
    \qquad\text{with}\qquad
    \|\vec{r}\|_2 \leq u g_n \|X\|_2 \|\vec{v}\|_2.
  \end{equation*}
  Let $\lambda = \sigma + 1/\theta$, with $\lambda = \infty$ if
  $\theta = 0$.  Then
  \begin{alignat}{3}
    \label{eq:eq:res_bound_large_eig}
    \| \theta A \vec{v} - (1+\sigma \theta) B\vec{v} \|_2 
    & \leq && u |1+\sigma \theta| \cdot |1-\sigma/\lambda|
    \Big( b_n + g_n  + a_n (1+2\max(1/\gamma, 1))  \Big) \nonumber\\
    & && \cdot K_2(A-\sigma B, B)\|B\|_2 \|\vec{v}\|_2 + O(u^2),
  \end{alignat}
  where we take $|1-\sigma/\lambda| = 1$ if $\lambda = \infty$ and
  $|1-\sigma/\lambda| = \infty$ if $\lambda = 0$ so that the bound is
  vacuous if $\lambda = 0$.  We also have
  \begin{alignat}{3}
    \label{eq:eq:res_bound_small_eig}
    \| \theta A \vec{v} - (1+\sigma \theta) B\vec{v} \|_2
    & \leq && u |\theta| \cdot |\hat{\sigma}| \cdot |1-\lambda/\sigma|
    \Big( b_n + g_n  + a_n (1+2\max(1/\gamma, 1))  \Big) \nonumber\\
    & && \cdot K_2(A-\sigma B, B) \|A\|_2 \|\vec{v}\|_2 
    + O(u^2).
  \end{alignat}
\end{theorem}
\begin{proof}
  With $G_0$ and $G_1$ as defined in the proof of Theorem~\ref{th:generalized_schur} we have
  \begin{equation*}
    (\breve{A}-G_0)X \vec{v} - \theta(A-\sigma B) \vec{v} = (A-\sigma B)\vec{r},
  \end{equation*}
  \begin{equation*}
    (B + G_1 - G_0 X)\vec{v}- \theta(A-\sigma B) \vec{v} = (A-\sigma B)\vec{r},
  \end{equation*}
  and
  \begin{equation*}
    \theta A \vec{v} - (1+\sigma \theta) B\vec{v} = G_1\vec{v} - G_0 X \vec{v} - (A-\sigma B)\vec{r}.
  \end{equation*}
  It follows that
  \begin{align}
    \| \theta A \vec{v} - (1+\sigma \theta) B\vec{v} \|_2 
    & \leq 
      \begin{aligned}[t] 
        u \Big( & b_n \|\breve{A}\|_2\|X\|_2 + a_n (\|\breve{A}\|_2
          + |\sigma| \|B\|_2) \|X\|_2 \\
        & + g_n \|A-\sigma B\|_2 \|X\|_2 \Big) \|\vec{v}\|_2
      \end{aligned} \nonumber\\
    & = 
      \begin{aligned}[t]
        u \Big( & (a_n+b_n+g_n) K_2(A-\sigma B, B) \|B\|_2 \\
                & + a_n |\hat{\sigma}| \|A\|_2 \|X\|_2\Big) \|\vec{v}\|_2 + O(u^2) 
      \end{aligned} \nonumber\\
    & = u \Big( b_n + g_n  + a_n (1+|\hat{\sigma}| / \gamma)  \Big) 
      K_2(A-\sigma B, B) \|B\|_2
      \|\vec{v}\|_2 \label{eq:res_proof_bound}.
  \end{align}
  We note that
  \begin{equation*}
    1+\sigma \theta 
    = 1 + \sigma \frac{1}{\lambda - \sigma} 
    = \frac{1}{1-\sigma/\lambda},
  \end{equation*}
  which together with \eqref{eq:res_proof_bound} and
  \eqref{eq:sigma0_gamma_bound} implies \eqref{eq:eq:res_bound_large_eig}.  With
  \begin{equation*}
    \theta 
    = \frac{1}{\lambda - \sigma} = \frac{\|B\|_2}{\hat{\sigma} \|A\|_2 (\lambda/\sigma - 1)}
  \end{equation*}
  we obtain
  \begin{equation*}
    \|B\|_2 = |\hat{\sigma}(1- \lambda/\sigma)| \|A\|_2,
  \end{equation*}
  which substituted into \eqref{eq:res_proof_bound} gives
  \eqref{eq:eq:res_bound_small_eig}.
\end{proof}

\section{Residual Bounds}
\label{sec:residual-bounds}



\bibliography{/home/mas/work/bib/ref}


\end{document}
