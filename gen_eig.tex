\documentclass{siamltex}
\usepackage[bookmarks=true]{hyperref}
% \usepackage{refcheck}
% \usepackage[margin=1.25in]{geometry}
\usepackage{mathtools}
\usepackage{amsmath}
% \usepackage{amsthm}
\usepackage{amsfonts}
\usepackage{bm}
\usepackage{amssymb}
\usepackage{bbold}
\usepackage{algorithm}
\usepackage[noEnd=false,indLines=false]{algpseudocodex}
\usepackage{colortbl}
\usepackage{multirow}
\usepackage{url}
\usepackage{placeins}
\usepackage{hhline}
% \usepackage{colonequals}
% \usepackage{tikz}
% \usepackage{pgfplots}
% \usepgfplotslibrary{fillbetween}
% \usepackage{graphicx}
\DeclareMathOperator*{\argmin}{arg\,min}
% \DeclareMathOperator{\diag}{diag}
\DeclareMathOperator{\real}{Re}
\DeclareMathOperator{\imag}{Imag}
\DeclareMathOperator{\fl}{fl}
\DeclareMathOperator{\sign}{sign}
% \DeclareMathOperator{\rank}{rank}
\DeclareMathOperator{\trace}{Tr}
\newcommand{\R}{\mathbb{R}}\newcommand{\C}{\mathbb{C}}\newcommand{\Z}{\mathbb{Z}}
\newcommand{\qed}{\rule{1.2ex}{1.2ex}}
\newcommand{\eqand}{\qquad\text{and}\qquad}
\newcommand{\eqor}{\qquad\text{or}\qquad}
\newcommand{\conj}[1]{\overline{#1}}
\renewcommand{\vec}[1]{\boldsymbol{#1}}
\newcommand{\T}{\mathrm{T}}
\renewcommand{\H}{\mathrm{H}}
\newcommand{\mcal}{\mathcal}
% \newtheorem{theorem}{Theorem}
% \newtheorem{lemma}{Lemma}
% \newtheorem{corollary}{Corollary}
\newtheorem{example}{Example}

% Add optional column specifiers for matrix environments.
\makeatletter
\renewcommand*\env@matrix[1][*\c@MaxMatrixCols c]{%
  \hskip -\arraycolsep
  \let\@ifnextchar\new@ifnextchar
  \array{#1}}
\makeatother

% Put paragraph indentation in enumerate environment.
% \let\oldenumerate=\enumerate
% \renewenvironment{enumerate}{\oldenumerate\parindent=1.5em}{\endlist}

\bibliographystyle{siam}
 \title{Pseudospectra and the Shift and Invert Solution
  of the Generalized Eigenvalue Problem}
\author{Michael Stewart\thanks{Department of Mathematics and
    Statistics, Georgia State University, Atlanta GA 30303, {\tt
      mastewart@gsu.edu}}}
\pagestyle{myheadings}
\thispagestyle{plain}
\markboth{MICHAEL STEWART}{SHIFT AND INVERT SOLUTION
  OF THE GENERALIZED EIGENVALUE PROBLEM}

\begin{document}
\maketitle
\begin{abstract}
  Given matrices $A$ and $B$, this paper gives error bounds for the solution of
  the generalized eigenvalue problem using a shift and invert strategy.  The
  analysis identifies circumstances under which the solution can be expected to
  satisfy useful error bounds.  The conditions are notably less restrictive than
  requiring that a shifted be well conditioned.
\end{abstract}
\begin{keywords}
  generalized eigenvalues, eigenvalues, eigenvectors, error analysis
\end{keywords}
\begin{AMS}
  15A18, 15A22, 15A23, 15A42, 65F15
\end{AMS}

\section{Background}
\label{sec:background}

In what follows we assume that $A$ and $B$ are $n\times n$ nonzero complex
matrices that are not necessarily hermitian.  Our goal is to solve the
generalized eigenvalue problem
\begin{equation}
  \label{eq:gen_eig1}
  A\vec{v}_i = \lambda B \vec{v}_i, \qquad \vec{v}_i \neq \vec{0}, \qquad
  i=1,2,\ldots n.
\end{equation}
We also use the formulation
\begin{equation}
  \label{eq:gen_eig2}
  \beta_i A \vec{v}_i = \alpha_i B \vec{v}_i, \qquad \vec{v}_i\neq \vec{0}, \qquad
  i = 1,2,\ldots, n.
\end{equation}
where $\beta_i$ and $\alpha_i$ are not both zero and
$\lambda_i = \alpha_i/\beta_i$.  If $\beta_i = 0$, we let $\lambda_i = \infty$.
If $A$ and $B$ have a common null space, then any common null vector would
satisfy \eqref{eq:gen_eig1} or \eqref{eq:gen_eig2} for any choice of $\lambda$
or $\alpha$ and $\beta$.  We therefore require that the intersection of the
null spaces of $A$ and $B$ be trivial, that is that the pencil $A-\lambda B$ be
\textit{regular}.  The problem is fully defined by the pair $(A,B)$ and the
generalized eigenvalues by $(\alpha_i, \beta_i)$.  When the context is clear,
we will often simply refer to eigenvalues and eigenvectors instead of
generalized eigenvalues and eigenvectors.  We also refer to both $\lambda_i$
and the pair $(\alpha_i, \beta_i)$ as eigenvalues.

If $B$ is nonsingular, then \eqref{eq:gen_eig1} can be converted to the
ordinary eigenvalue problem by solving the ordinary eigenvalue problem
$B^{-1}A$, which in practice would be computed using some stable factorization
of $B$.  If $A$ is nonsingular, then each eigenvalue of $A^{-1}B$ is of the
form $1/\lambda_i$, where $\lambda_i$ satisfies (\ref{eq:gen_eig1}).  Either
formulation can be used to solve (\ref{eq:gen_eig1}).  However, if the matrix
to be inverted is ill conditioned then the computed $B^{-1}A$ (or $A^{-1}B$)
might have large errors.  In this case, the computed generalized eigenvalues
might be inaccurate, even when they are well conditioned \cite{govl:13,
  stew:01}.

To try to avoid problems with ill conditioning of $A$ when forming $A^{-1}B$,
we can choose a shift for which $A-\sigma B$ is nonsingular and instead form
\begin{equation*}
X = (A-\sigma B)^{-1} B.
\end{equation*}
If
\begin{equation}
  \label{eq:X_problem}
  X \vec{v}_i = \theta_i \vec{v}_i, \qquad \vec{v}_i \neq 0, 
  \qquad i=1,2,\ldots, n,
\end{equation}
then (\ref{eq:gen_eig2}) holds with
\begin{equation*}
  \alpha_i = 1+\sigma \theta_i, \eqand \beta_i = \theta_i.
\end{equation*}
We then have
\begin{equation*}
  \lambda_i = (1+\sigma \theta_i)/\theta_i = \frac{1}{\theta_i} + \sigma
\end{equation*}
in (\ref{eq:gen_eig1}).  Note that either $\alpha_i$ or $\beta_i$ must be
nonzero.  This paper focuses on the numerical properties of the eigenvalue
problem (\ref{eq:X_problem}) as a means for solving (\ref{eq:gen_eig2}).
However, the error bounds depend on the eigenvalues $\lambda_i$ of
(\ref{eq:gen_eig1}) and on properties of the matrix $B^{-1}A$.\footnote{If $B$
  is nonsingular.  In\S~\ref{sec:pseudospectra} and \S\ref{sec:pseud-bounds-k2}
  we go to some effort to formulate theorems that also apply when $B$ is
  singular.}

Unfortunately, it is not always possible to choose $\sigma$ so that
$A-\sigma B$ is well conditioned.  Even with the use of a shift, the explicit
conversion of the generalized eigenvalue problem to an ordinary eigenvalue
problem is not commonly recommended, except perhaps in the use of iterative
methods like the Arnoldi algorithm.  The $QZ$ algorithm \cite{most:73} instead
solves the generalized eigenvalue by computing orthogonal $Q$ and $Z$ for which
\begin{equation*}
  A = Q T_a Z^\H, \eqand B = QT_b Z^\H
\end{equation*}
where $T_a$ and $T_b$ are upper triangular.  If the $i$th diagonal elements of
$T_a$ and $T_b$ are $\alpha_i$ and $\beta_i$ respectively, then each
$(\alpha_i, \beta_i)$ is an eigenvalue of \eqref{eq:gen_eig2}.  The
decomposition is perfectly backward stable so that for $i=1,\ldots, n$ the
pairs $(\alpha_i, \beta_i)$ form the set of eigenvalues of a pair of matrices
close to $(A,B)$.  The generalized eigenvectors can be obtained from $Z$ in a
manner analogous to what is done with a Schur decomposition for the ordinary
eigenvalue problem.  The only disadvantage of this method for a general dense
problem is the computational cost of the $QZ$ algorithm relative to that of
solving an ordinary eigenvalue problem for $(A-\sigma B)^{-1}B$.  However,
given the possible instability associated with forming $(A-\sigma B)^{-1}B$,
the $QZ$ algorithm is the standard method for the dense nonhermitian
generalized eigenvalue problem.

In this paper, we revisit the use of inversion to convert the generalized
eigenvalue problem to an ordinary eigenvalue problem and show that, while not
backward stable, it can be more stable than consideration of the condition
number of $A-\sigma B$ might suggest.  Further, the instability is detectable
prior to computing an eigenvalue decomposition of $X$ and can be quantified in
backward error and residual bounds.  In particular, if we define
\begin{equation}
  \label{eq:K2_C_def}
  K_2(C, B) = \frac{\|C\|_2 \|C^{-1} B\|_2}{\|B\|_2}
\end{equation}
then we will see in the error bounds that
\begin{equation}
  \label{eq:K2_def}
  K_2(A-\sigma B, B) = \frac{\|A-\sigma B\|_2}{\|B\|_2} \|(A-\sigma B)^{-1} B\|_2
\end{equation}
is the key factor affecting the stability, or lack of stability, of the shift
and invert approach.  Note that if $B$ is nonsingular then
$\|(A-\sigma B)^{-1}B\|_2= \|(B^{-1}A-\sigma I)^{-1}\|_2$ is the resolvent norm
of $B^{-1}A$, which immediately suggests a connection to the pseudospectrum of
$B^{-1}A$.

If we define a scaled shift and scaled eigenvalues
\begin{equation}
  \label{eq:scaled_eig_shift}
  \hat{\sigma} = \frac{\|B\|_2}{\|A\|_2} \sigma, \qquad
  \hat{\lambda}_i = \frac{\|B\|_2}{\|A\|_2} \lambda_i, \qquad
  i = 1, 2,\ldots, n,
\end{equation}
and
\begin{equation*}
  \hat{A} = \frac{A}{\|A\|_2}, \eqand
  \hat{B} = \frac{B}{\|B\|_2},
\end{equation*}
then the condition number of $A-\sigma B$ satisfies $\kappa_2(A-\sigma B) = \kappa_2(\hat{A} - \hat{\sigma} \hat{B})$.
For $K_2(A-\sigma B, B)$, we have the similar identity
\begin{equation*}
  K_2(A-\sigma B, B)
  = \|\hat{A} - \hat{\sigma} \hat{B}\|_2
  \|(\hat{A}-\hat{\sigma} \hat{B})^{-1} \hat{B}\|_2
  = K_2(\hat{A}-\hat{\sigma} \hat{B}, \hat{B}).
\end{equation*}
Thus $K_2(A- \sigma B, B)$ does not change with scaling of $A$ and $B$, so long as
the shift is scaled accordingly.

It is shown in \S~\ref{sec:pseud-bounds-k2} that
\begin{equation}
  \label{eq:upper_lower_bounds}
  \frac{\|\hat{A}-\hat{\sigma} \hat{B}\|_2}{\min_i |\hat{\lambda_i} - \hat{\sigma}|}
  \leq K_2(A-\sigma B,B) 
  \leq \kappa_2(\hat{A}-\hat{\sigma} \hat{B}),
\end{equation}
where the minimum is over all scaled generalized eigenvalues $\hat{\lambda}_i$
of the pair $(A,B)$, assuming $1/|\hat{\lambda}_i - \hat{\sigma}| = 0$ if
$\lambda_i = \infty$.  The upper bound is tight when
$\|(A-\sigma B)^{-1} B\|_2 = \|(A-\sigma B)^{-1} \|_2 \|B\|_2$.  The lower
bound will be seen to be an equality if $(A,B)$ has a complete orthonormal set
generalized eigenvectors.  If $B$ is nonsingular, this is equivalent to
$B^{-1}A$ being normal.  In the case in which $(A,B)$ has orthonormal
eigenvectors, it is often not too difficult to select a shift for which
$K_2(A - \sigma B, B)$ is not large.


The structure of the paper is as follows.  In \S\ref{sec:pseudospectra} we
consider a definition of the pseudospectrum of the pair $(A,B)$.  In
\S\ref{sec:pseud-bounds-k2} we give a characterization of $K_2(A-sigma B,B)$ in
terms of the pseudospectrum of $(A,B)$.  In \S\ref{sec:algor-error-analys} we
give error bounds for an algorithm that uses shift and invert to compute a
generalized Schur decomposition of $(A,B)$.  In \S\ref{sec:residual-bounds} we
present bounds that relate residuals for the ordinary eigenvalue problem
(\ref{eq:X_problem}) to residuals for the generalized eigenvalue problem
(\ref{eq:gen_eig2}).  The magnitude of $K_2(A-\sigma B, B)$ is the key measure
of instability in both \S\ref{sec:algor-error-analys} and
\S\ref{sec:residual-bounds}.

\section{Pseudospectra of $(A,B)$}
\label{sec:pseudospectra}

In this section, we consider the choice of shift and its impact on
$K_2(A-\sigma B, B)$.  A rough summary of where we will end up is that we can
expect to have useful error bounds when $\sigma$ is not in the
$\epsilon$-pseudospectrum of $B^{-1}A$ for small $\epsilon$.  However, we do
not want to assume that $B$ is nonsingular.  We begin by modifying some
standard results on pseudospectra to accommodate singular $B$.

In addition to just looking at the pseudospectrum of $B^{-1}A$ when $B$ is
nonsingular, there are several of generalizations of pseudospectra to matrix
pencils.  See \cite{trem:05} for a survey.  Most of these assume that $B$ is
nonsingular.  Notable exceptions include the approaches of \cite{hiti:02} and
\cite{emke:17}.  The former uses a definition that allows independent
perturbation of $A$ and $B$.  The latter focuses on applications to
differential algebraic equations and uses a Schur factorization to isolate zero
eigenvalues of $(A-\sigma B)^{-1}B$, corresponding to infinite eigenvalues of
$(A,B)$, in a nilpotent block in order to remove them from consideration.
Neither approach seems well aligned with our goal of understanding the effect
of $\sigma$ on the bounds to be proven in \S\ref{sec:algor-error-analys} and
\S\ref{sec:residual-bounds}.  We instead consider a definition of the
pseudospectrum that includes infinite eigenvalues and exactly matches the
pseudospectrum of $B^{-1}A$ when $B$ is nonsingular.

We assume throughout that $(A,B)$ is a regular pair.  For
$\widehat{\C} = \C\cup \{\infty\}$, let $\Lambda_0(A,B)\subseteq \widehat{\C}$
denote the set of generalized eigenvalues of $(A,B)$.  For $\epsilon > 0$, we
define the $\epsilon$-pseudospectrum of $(A,B)$ by
\begin{equation}
  \label{eq:pseudo_def}
  \Lambda_\epsilon(A,B) = \left\{ \sigma \in \widehat{\C} :
    \text{$\sigma \in \Lambda_0(A-BE, B)$ for some $E$
      with $\|E\|_2 < \epsilon$} \right\}.
\end{equation}
When $B$ happens to be nonsingular, this is easily seen to be the
$\epsilon$-pseudospectrum of $B^{-1}A$.  The pair $(A-BE, B)$ has infinite
generalized eigenvalues if and only if $B$ is singular and, hence, if and only
if $(A, B)$ has infinite eigenvalues.  It follows that for any $\epsilon > 0$,
$\infty \in \Lambda_\epsilon(A,B)$ if and only if $\infty \in \Lambda_0(A,B)$.

Let
\begin{equation*}
  \mcal{E}(A,B,\sigma) = \{E : \sigma \in \Lambda_0(A- BE, B)\}.
\end{equation*}
For $\sigma \in \widehat{\C}$, we can define
\begin{equation}
  \label{eq:delta_def}
  \delta(A, B, \sigma) 
  = \inf_{E\in \mcal{E}(A,B,\sigma)} \|E\|_2,
\end{equation}
with the convention that if the set $\mcal{E}(A,B,\sigma)$ is empty, then
$\delta(A, B, \sigma) = \infty$.  It is easily seen that $\delta(A,B,\infty)=0$
if $B$ is singular and $\delta(A,B,\infty) = \infty$ if $B$ is nonsingular.
For brevity in longer equations, we sometimes assume that $A$ and $B$ are given
and let $\delta(\sigma) = \delta(A, B, \sigma)$.  We refer to
$\delta(A,B,\sigma)$ as the \textit{pseudospectral separation} of $\sigma$ from
$\Lambda_0(A,B)$.

We will show in the proof of Lemma~\ref{lm:X_norm_bound} that with $B\neq 0$
and $\sigma \in \C$, $\mcal{E}(A,B,\sigma)$ is nonempty and there exists
$E_0\in \mcal{E}(A,B,\sigma)$ for which $\delta(A,B,\sigma)=\|E_0\|_2 <\infty$.
Assuming this for the moment, we have the following lemma.
\begin{lemma}
  \label{lm:delta_pseudo}
  If $(A,B)$ is regular, then
  \begin{equation}
    \label{eq:pseudo-def-delta}
    \Lambda_\epsilon(A,B) = \left\{ \sigma \in \widehat{\C} : \delta(A,B,\sigma) < \epsilon\right\}.
  \end{equation}
\end{lemma}
\begin{proof}
  If $B\neq 0$, $\sigma\in \C$, and $\delta(A,B,\sigma)< \epsilon$, then there
  exists $E_0$ with $\|E_0\|_2 < \epsilon$ for which
  $\sigma \in \Lambda_0(A-BE_0, B)$ so that $\sigma$ is in
  $\Lambda_\epsilon(A,B)$ as defined in \eqref{eq:pseudo_def}.  The matrix
  $E_0$ is constructed using a generalized SVD in the proof of
  Lemma~\ref{lm:X_norm_bound} below.  Conversely, if
  $\delta(A,B,\sigma)\geq \epsilon$, then $\|E\|_2 \geq \epsilon$ for all $E$
  with $\sigma\in \Lambda_0(A-BE, B)$.  Thus, for $B\neq 0$ and $\sigma\in\C$,
  $\sigma$ is in the $\epsilon$-pseudospectrum of $(A,B)$ if and only
  $\delta(A,B,\sigma) < \epsilon$.

  If $B=0$, then regularity requires that $A$ be nonsingular.  For
  $\sigma \in \C$ and $B=0$, it follows that $\mcal{E}(A,B,\sigma)$ is empty,
  since $\sigma \in \Lambda_0(A-BE, B)$ would require
  $(A- BE)\vec{x} = \sigma B\vec{x}$ or $A\vec{x} =\vec{0}$ for
  $\vec{x}\neq \vec{0}$.  So by definition
  \eqref{eq:pseudo_def}$, \Lambda_\epsilon(A,B)$ contains no $\sigma \in \C$.
  In this case, since $\mcal{E}(A,B,\sigma)$ is empty,
  $\delta(A,B,\sigma)=\infty$ and \eqref{eq:pseudo-def-delta} also gives no
  $\sigma\in\C$ in $\Lambda_\epsilon(A,B)$.  Thus, for all $\sigma \in \C$,
  \eqref{eq:pseudo_def} and \eqref{eq:delta_def} include the same values.


  We have already noted in relation to \eqref{eq:pseudo_def} that
  $\infty\in \Lambda_\epsilon(A,B)$ if and only if $B$ is singular.  If $B$ is
  singular, then $\delta(A,B,\infty) = 0$ so that \eqref{eq:pseudo-def-delta}
  also implies that $\infty \in \Lambda_\epsilon(A,B)$.  If $B$ is nonsingular,
  then $\mcal{E}(A,B,\infty)$ is empty and $\delta(A,B,\infty) = \infty$, so
  that \eqref{eq:pseudo-def-delta} gives $\infty \notin \Lambda_\epsilon(A,B)$.
  Thus, for all $\sigma \in \widehat{\C}$, \eqref{eq:pseudo_def} and
  \eqref{eq:delta_def} include the same values.
\end{proof}

If $B$ is nonsingular and $\sigma$ is not an eigenvalue of $B^{-1}A$, then it
is well known that $\sigma$ is in the $\epsilon$-pseudospectrum of $B^{-1}A$ if
and only if $\|(B^{-1}A - \sigma I)^{-1}\|_2 > 1/\epsilon$.  This is equivalent
to $\delta(A, B, \sigma) = 1/\|(B^{-1}A - \sigma I)^{-1}\|_2$.  The following lemma
reformulates this result to avoid assuming invertibility of $B$.
\begin{lemma}
  \label{lm:X_norm_bound}
  Let $(A,B)$ be regular.  With $\delta(A, B, \sigma)$ defined as in
  \eqref{eq:delta_def}, we have
  \begin{equation*}
     \delta(A, B, \sigma) 
     = 
     \begin{cases}
       \frac{1}{\|(A-\sigma B)^{-1} B\|_2}, 
       & \text{$\sigma\in\C$ and $\sigma \notin \Lambda_0(A,B)$}, \\
       \infty, & \text{$\sigma=\infty$ and $\infty \notin \Lambda_0(A,B)$}, \\
       0, & \sigma \in \Lambda_0(A,B), \\
       \end{cases}
  \end{equation*}
  where in the first case we let $1/\|(A-\sigma B)^{-1} B\|_2 = \infty$ if $B=0$.
\end{lemma}
\begin{proof}
  We assume the conditions of the first case.  If $B=0$, then
  $\mcal{\mcal{E}}(A,B,\sigma)$ is empty and $\delta(A,B,\sigma) = \infty$.
  Suppose that $B\neq 0$.  The matrix $A-\sigma B$ is nonsingular and we
  consider the generalized singular value decomposition
  \begin{equation*}
    A - \sigma B = S I U^\H,\eqand
    B = S D V^\H,
  \end{equation*}
  where $S=(A-\sigma B)U$ is invertible, $(A-\sigma B)^{-1} B = U D V^\H$, $U$
  and $V$ are unitary, $D = \diag(d_1, \ldots, d_n)$, $d_1 > 0$, and
  $d_k\geq 0$ is nonincreasing for $k=1,2,\ldots, n$.  Clearly
  $A-\sigma B - BE = S (I - D V^\H E U) U^\H$ is singular if and only if
  $(I - D V^\H E U)$ is singular which implies that $\|E\|_2 \geq 1/\|D\|_2$ or
  that $\|E\|_2 \geq 1/d_1$ for all $E\in\mcal{E}(A,B,\sigma)$.  Thus
  \begin{equation*}
    \delta(A,B,\sigma)
    = \inf_{E \in \mcal{E(A,B,\sigma)}} \|E\|_2
    \geq \frac{1}{d_1}
    = \frac{1}{\|(A-\sigma B)^{-1}B\|_2}.
  \end{equation*}
  Let  
  \begin{equation*}
    E_0 = \frac{1}{d_1}V \vec{e}_1 \vec{e}_1^\H U^\H.
  \end{equation*}
  Then
  \begin{equation*}
    A-\sigma B - B E_0 = S (I-\vec{e}_1 \vec{e}_1^\H) U^\H
  \end{equation*}
  is singular so that
  \begin{equation*}
    \delta(A,B,\sigma)
    \leq \|E_0\|_2 
    = \frac{1}{d_1} = \frac{1}{\|(A-\sigma B)^{-1} B\|_2}.
  \end{equation*}
  This establishes the first case.

  For the second case, the second condition implies that $B$ is nonsingular so
  that $\mcal{(A,B,\infty)}$ is empty and $\delta(A,B,\sigma) = \infty.$ The
  third case implies that $E=0$ is in $\mcal{E}(A,B,\sigma)$ so that
  $\delta(A,B,\sigma)=0$.
\end{proof}


If $B^{-1}A$ is normal then it is easily seen that
$\delta(\sigma) = \min_i |\lambda_i - \sigma|$, where the minimum is taken over
all generalized eigenvalues $\lambda_i$ of $(A,B)$.  This also generalizes to
the case in which $B$ is singular.
\begin{lemma}
  \label{lm:normal_eig_bound}
  Assume that $(A,B)$ is a regular pair with an orthonormal basis of
  generalized eigenvectors $\vec{v}_i$, $i = 1, \ldots, n$. Then for
  $\sigma \in \widehat{C}$ we have
  \begin{equation}
    \label{eq:normal_delta}
    \delta(A,B,\sigma) = \min_i |\lambda_i - \sigma|,
  \end{equation}
  where we assume $|\lambda_i - \sigma| = 0$ if $\lambda_i=\sigma=\infty$ and
  $|\lambda_i - \sigma| = \infty$ if exactly one of $\lambda_i$ or $\sigma$
  equals $\infty$.
\end{lemma}
\begin{proof}
  We first consider $\sigma = \infty$.  The claim follows immediately with
  $\delta(A,B,\sigma)=0$ if $\sigma$ is a generalized eigenvalue of $(A,B)$,
  including the case of $\sigma = \infty$.  We have already noted that
  $\delta(A,B,\infty)=\infty$ when $B$ is nonsingular and $\infty$ is not a
  generalized eigenvalue of $(A,B)$.  Both of these cases are consistent with
  \eqref{eq:normal_delta}.
  

  We consider the more interesting case $\sigma\in\C$.  If $B=0$, then $(A,B)$
  has no finite generalized eigenvalues and $\delta(A,B,\sigma)=\infty$, which
  is consistent with \eqref{eq:normal_delta}.  As above, if
  $\sigma\in\Lambda_0(A)$, \eqref{eq:normal_delta} follows with
  $\delta(A,B,\sigma)=0$.  So we assume that $B\neq 0$ and that $A-\sigma B$ is
  nonsingular.  Let $V$ be the unitary matrix with $i$th column $\vec{v}_i$.
  For each generalized eigenvector we have
  $\beta_i A\vec{v}_i = \alpha_i B \vec{v}_i$, where regularity ensures that we
  do not have both $A\vec{v}_i=\vec{0}$ and $B\vec{v}_i= \vec{0}$.  It follows
  that the set $\{A\vec{v}_i, B\vec{v}_i\}$ spans a subspace of dimension one.
  Let $\vec{s}_i\neq \vec{0}$ be in this space and let $S$ be the matrix with
  $i$th column $\vec{s}_i$.  We then have
  \begin{equation*}
    AV = S D_a, \eqand
    BV = S D_b
  \end{equation*}
  for diagonal matrices $D_a$ and $D_b$.  Thus
  \begin{equation*}
    (A-\sigma B) V = S (D_a - \sigma D_b).
  \end{equation*}
  Since $A-\sigma B$ is nonsingular and $V$ is unitary, $S$ must be
  nonsingular.  Let the diagonal elements of $D_a$ be $\hat{\alpha}_i$ and
  those of $D_b$ be $\hat{\beta}_i$ for $i=1,\ldots, n$.  We have
  \begin{align*}
    A - \sigma B - BE 
    & = S (D_a - \sigma D_b - D_b V^\H E V) V^\H \\
    & = S (D_a - \sigma D_b)\left( I - (D_a - \sigma D_b)^{-1} D_b V^\H E V \right) V^\H.
  \end{align*}
  This is singular if and only if $I - (D_a - \sigma D_b)^{-1} D_b V^\H E V$ is
  singular which implies
  \begin{equation*}
    \|(D_a - \sigma D_b)^{-1} D_b\|_2 \|V^\H E V\|_2 \geq \|(D_a - \sigma D_b)^{-1} D_b V^\H E V\|_2 \geq 1.
  \end{equation*}
  We have
  \begin{equation*}
    (D_a - \sigma D_b)^{-1} D_b = \diag\left(\frac{1}{(\lambda_1-\sigma)},
      \ldots, \frac{1}{(\lambda_n-\sigma)}\right)
  \end{equation*}
  so that
  \begin{equation}
    \label{eq:E_lower_bound}
    \|E\|_2 \geq\frac{1}{\|(D_a - \sigma D_b)^{-1} D_b\|_2} 
    = \frac{1}{\max_i 1/|\lambda_i - \sigma|} = \min_i |\lambda_i - \sigma|.
  \end{equation}
  The pair $(A,B)$ has a finite generalized eigenvalue for each $\hat{\beta}_i$
  satisfying $\hat{\beta}_i\neq 0$.  Since we have assumed that
  $B = S D_b V^\H\neq 0$, we must have at least one nonzero $\hat{\beta}_i$ so
  that $\min_i |\lambda_i - \sigma|$ is finite.  Without loss of generality,
  assume that the columns of $V$ and $S$ are permuted so that
  $|\lambda_1 - \sigma| = \min_i |\lambda_i - \sigma|$ and
  $\hat{\beta}_1\neq 0$.  Then
  \begin{equation*}
    E = \frac{\hat{\alpha}_1 - \sigma \hat{\beta}_1}{\hat{\beta}_1} V \vec{e}_1 \vec{e}_1^\T V^\H =
    (\lambda_1 - \sigma) V \vec{e}_1 \vec{e}_1^\T V^\H
  \end{equation*}
  ensures that $ I - (D_a - \sigma D_b)^{-1} D_b V^\H E V$ is singular with
  $\vec{e}_1$ as a null vector.  Hence $A-\sigma B - BE$ is singular and, since
  it achieves the lower bound (\ref{eq:E_lower_bound}), this choice of $E$ also
  has minimal norm with $\delta(A,B,\sigma) = \|E\|_2$ satisfying
  \eqref{eq:normal_delta}.
\end{proof}

\begin{lemma}
  \label{lm:bauer_fike_bound}
  Let $(A,B)$ regular.  Assume that the columns of $V$ give a complete set of
  eigenvectors for $(A,B)$.  Then
  \begin{equation}
    \label{eq:bauer-fike-bound}
    \frac{1}{\delta(A,B,\sigma)}
    \leq \kappa_2(V) \frac{1}{\min_i |\lambda_i - \sigma|},
  \end{equation}
  where we make the same assumptions as Lemma~\ref{lm:normal_eig_bound} about
  $\min_i |\lambda_i - \sigma|$ when either $\lambda_i=\infty$ or
  $\sigma = \infty$.
\end{lemma}
\begin{proof}
  The case $\sigma = \infty$ is the same as in the proof of
  Lemma~\ref{lm:normal_eig_bound}, with $\delta(A,B,\sigma) = 0$ or
  $\delta(A,B\sigma)=\infty$, depending on whether $(A,B)$ has an infinite
  eigenvalue.  The case $B=0$ and $\sigma\in \Lambda_0(A,B)$ are also similar.
  So we assume that $\sigma\in\C$, $B\neq 0$, and $A-\sigma$ is singular.  The
  rest of the proof is also similar to the first part of
  Lemma~\ref{lm:bauer_fike_bound}.  Let
  $\beta_i A \vec{v}_i = \alpha_i B \vec{v_i}$ with $\alpha_i$ and $\beta_i$
  not both zero.  We can construct a nonsingular $S$ such that $A V = SD_a$ and
  $BV= SD_b$, where the diagonal elements of $D_a$ are $\hat{\alpha}_i$ and
  those of $D_b$ are $\hat{\beta}_i$.  We then have
  \begin{multline*}
    \|(A-\sigma B)^{-1} B\|_2
    = \|V (D_a - \sigma D_b)^{-1} D_b V^{-1}\|_2
    \leq \kappa_2(V) \|(D_a - \sigma D_b)^{-1} D_b \|_2 \\
    = \kappa_2(V) \frac{1}{\min_i | \lambda_i - \sigma|}.
  \end{multline*}
\end{proof}

% Finally, we can bound $K_2(A-\sigma B, B)$ in terms of the condition number of
% the matrix of generalized eigenvectors.
% \begin{lemma}
%   \label{lm:bauer_fike_bound}
%   Let $B\neq 0$ and $\sigma \notin \Lambda_0$.  Assume that the columns of $V$ give a
%   complete set of eigenvectors for $(A,B)$.  Then
%   \begin{equation*}
%     K_2(A-\sigma B, B) 
%     \leq \kappa_2(V) \frac{\|\hat{A} - \hat{\sigma} \hat{B}\|_2}
%     {\min_i |\hat{\lambda}_i - \hat{\sigma}|}
%     = \kappa_2(V) \frac{1+|\hat{\sigma}|}{\min_i |\hat{\lambda}_i - \hat{\sigma}|}
%     = \kappa_2(V) \frac{1+1/|\hat{\sigma}|}{\min_i |1 - \hat{\lambda}_i / \hat{\sigma}|}.
%   \end{equation*}
% \end{lemma}
% \begin{proof}
%   The proof uses the same approach as the start of the proof of
%   Lemma~\ref{lm:normal_eig_bound}.  Let
%   $\beta_i A \vec{v}_i = \alpha_i B \vec{v_i}$ with $\alpha_i$ and $\beta_i$ no
%   both zero.  As in Lemma~\ref{lm:normal_eig_bound} we can construct an $S$
%   such that $A V = SD_a$ and $BV= SD_b$ where the diagonal elements of $D_a$
%   are $\hat{\alpha}_i$ and those of $D_b$ are $\hat{\beta}_i$.  We then have
%   \begin{multline*}
%     \|(A-\sigma B)^{-1} B\|_2
%     = \|V (D_a - \sigma D_b)^{-1} D_b V^{-1}\|_2
%     \leq \kappa_2(V) \|(D_a - \sigma D_b)^{-1} D_b \|_2 \\
%     = \kappa_2(V) \frac{1}{\min_i | \lambda_i - \sigma|}.
%   \end{multline*}
%   The bounds involving scaled matrices and quantities are given by the same
%   manipulation we have used in \eqref{eq:upper_bound_for_lower_bound}.
% \end{proof}

\section{Pseudospectral Bounds on $K_2(A,B)$}
\label{sec:pseud-bounds-k2}

We revisit the quantity $K_2(A-\sigma B, B)$ introduced in
\S\ref{sec:background}, starting by proving the bounds
\eqref{eq:upper_lower_bounds}.
\begin{lemma}
  The quantity $K_2(A-\sigma B, B)$ satisfies (\ref{eq:upper_lower_bounds}).
\end{lemma}
\begin{proof}
  The upper bound is immediate from
  \begin{multline*}
    K_2(A-\sigma B, B)
    = \frac{\|A-\sigma B\|_2}{\|B\|_2} \|(A-\sigma B)^{-1} B\|_2
    \leq \frac{\|A-\sigma B\|_2}{\|B\|_2} \|(A-\sigma B)^{-1}\|_2 \|B\|_2 \\
    = \|A-\sigma B\|_2 \|(A-\sigma B)^{-1}\|_2
    = \|\hat{A}-\hat{\sigma} \hat{B}\|_2 \|(\hat{A}-\hat{\sigma} \hat{B})^{-1}\|_2.
  \end{multline*}
  For the lower bound, we let $A = Q T_a Z^\H$ and $B = Q T_b Z^\H$ be a
  generalized Schur decomposition of $A$ and $B$, where the diagonal elements
  of $T_a$ are $\alpha_i$ and the diagonal elements of $T_b$ are $\beta_i$.  Then
  \begin{equation*}
    \|(A-\sigma B)^{-1} B\|_2 = \|T\|_2, \qquad T = (T_a - \sigma T_b)^{-1} T_b.
  \end{equation*}
  We then have
  \begin{equation*}
    \|(A-\sigma B)^{-1} B\|_2 
    \geq \rho(T) 
    = \max_i \left| \frac{\beta_i}{\alpha_i - \sigma \beta_i} \right|
    = \max_i \frac{1}{|\lambda_i - \sigma|}
    = \frac{1}{\min_i |\lambda_i - \sigma|},
  \end{equation*}
  where $\rho(T)$ is the spectral radius of $T$.  It should be understood that
  if $\beta_i=0$, then $\lambda_i = \infty$ and $1/|\lambda_i -\sigma|=0$.  With scaling, we have
  \begin{equation*}
    K_2(A-\sigma B, B)
    \geq \frac{\|A-\sigma B\|_2}{\|A\|_2} 
     \frac{1}{\min_i \|B\|_2 |\lambda_i - \sigma|/\|A\|_2}
     = \|\hat{A} - \hat{\sigma} \hat{B}\|_2
     \frac{1}{\min_i |\hat{\lambda}_i - \hat{\sigma}|}.
  \end{equation*}
\end{proof}

To interpret the lower bound, we define
\begin{equation}
  \label{eq:gamma_def}
  \gamma = \frac{\|A-\sigma B\|_2}{\|A\|_2}
  = \|\hat{A}-\hat{\sigma} \hat{B}\|_2 \leq 1+|\hat{\sigma}|
\end{equation}
in the lower bound is large only if the scaled shift $\hat{\sigma}$ is large.
This suggests that to moderate $K_2(A-\sigma B, B)$, we might need to limit the
magnitude of the shift.  However, for large shift we can note that
\begin{equation}
  \label{eq:upper_bound_for_lower_bound}
  \frac{\|\hat{A}-\hat{\sigma} \hat{B}\|_2}{\min_i |\hat{\lambda_i} - \hat{\sigma}|}
  \leq \frac{1+|\hat{\sigma}|}{\min_i |\hat{\lambda_i} - \hat{\sigma}|}
  = \frac{1+ 1/|\hat{\sigma}|}{\min_i |1 - \hat{\lambda_i}/\hat{\sigma}|}
  = \frac{1+ 1/|\hat{\sigma}|}{\min_i |1 - \lambda_i/\sigma|}
\end{equation}
so that for larger scaled shifts, the magnitude of the lower bound is
constrained by the relative distance of $\sigma$ from a generalized eigenvalue.
For small scaled shifts, the absolute distance is more important.

We return to the quantity $K_2(A-\sigma B, B)$ defined in (\ref{eq:K2_def}).
From Lemma~\ref{lm:X_norm_bound} we know that
\begin{equation*}
  K_2(A-\sigma B, B) = \frac{\|A-\sigma B\|_2}{\|B\|_2} \frac{1}{\delta(\sigma)}.
\end{equation*}
The scaling in this expression is inconvenient.  To fix this, we note that
\begin{align*}
  K_2(A-\sigma B, B) 
  & = \frac{\|A-\sigma B\|_2}{\|B\|_2} \|(A-\sigma B)^{-1}B\|_2 \\
  & = \left( \frac{\|A-\sigma B\|_2}{\|B\|_2} \frac{\|B\|_2}{\|A\|_2}\right) 
    \left( \frac{\|A\|_2}{\|B\|_2}\|(A-\sigma B)^{-1}B\|_2  \right)\\
  & = \left\| \frac{A}{\|A\|_2} - \sigma \frac{\|B\|_2}{\|A\|_2}
    \frac{B}{\|B\|_2}
    \right\|_2 \left\|\left(\frac{A}{\|A\|_2}
    - \sigma \frac{\|B\|_2}{\|A\|_2} \frac{B}{\|B\|_2}\right)^{-1}
    \frac{B}{\|B\|_2}\right\|_2 \\
  & = \|\hat{A} - \hat{\sigma} \hat{B}\|_2 
    \|(\hat{A} - \hat{\sigma} \hat{B})^{-1} \hat{B}\|_2
\end{align*}
where $\hat{A} = A/\|A\|_2$, $\hat{B} = B/\|B\|_2$, and $\hat{\sigma}$ is the
scaled shift in (\ref{eq:scaled_eig_shift}).  This suggests a scaled
version of $\delta(\sigma)$, so we define
\begin{equation*}
  \hat{\delta}(\hat{\sigma}) 
  = \|(\hat{A} - \hat{\sigma} \hat{B})^{-1} \hat{B}\|_2
  = \frac{\|B\|_2}{\|A\|_2}\frac{1}{\|(A-\sigma B)^{-1}B\|_2}
  = \frac{\|B\|_2}{\|A\|_2} \delta(\sigma).
\end{equation*}
We then have
\begin{equation}
  \label{eq:K2_delta_ident}
  K_2(A-\sigma B, B) 
  = \|\hat{A} - \hat{\sigma} \hat{B}\|_2 \frac{1}{\hat{\delta}(\hat{\sigma})}
  \leq \frac{1+|\hat{\sigma}|}{\hat{\delta}(\hat{\sigma})}
  = \frac{1+1/|\hat{\sigma}|}{\hat{\delta}(\hat{\sigma})/|\hat{\sigma}|}
\end{equation}
where the final identity assumes that $\hat{\sigma} \neq 0$.  The two forms of
the upper bound in \eqref{eq:K2_delta_ident} are analogous to
\eqref{eq:upper_bound_for_lower_bound}.  For larger values of $\hat{\sigma}$,
$K_2(A-\sigma B, B)$ remains moderate if $\hat{\delta}(\hat{\sigma})$ is not
too small relative to $|\hat{\sigma}|$.


\section{Algorithm and Error Analysis}
\label{sec:algor-error-analys}

We now turn our attention to the computation of a generalized Schur
decomposition of $(A,B)$ from a schur decomposition of
$X = (A-\sigma B)^{-1} B$.  Suppose that $\sigma \notin \Lambda_0$
and that we have a Schur decomposition $X = ZTZ^\H$.  We then have
\begin{equation*}
  (A -\sigma B) Z T = B Z,\qquad\mbox{or}\qquad
  A Z T = B Z (I +\sigma T).
\end{equation*}
If we compute a $QR$ factorization $(A -\sigma B)Z = QT_1$, then $T_1$
is nonsingular and $Q^\H (A-\sigma B) Z = T_1$ so that
$Q^\H BZ = T_1 T$.  This leads to a generalized Schur decomposition
\begin{equation}
  \label{eq:Gen_schur_decomp}
  Q^\H A Z = T_1(I+\sigma T),\eqand Q^\H BZ = T_1 T.
\end{equation}
Since $T_1$ is nonsingular, the generalized eigenvalues of $(A,B)$ are the same
as those of $(I+\sigma T, T)$.  If the diagonal elements of $T$ are $\theta_i$
for $i=1,\ldots n$, then the eigenvalues of $(A,B)$ are
$(1+\sigma\theta_i, \theta_i)\neq (0,0)$.  The finite generalized eigenvalues
are $\lambda_i = (1+\sigma \theta_i)/\theta_i$ for each $\theta_i\neq 0$.  The
first $k$ columns of $Z$ give an orthonormal basis for a deflating subspace of
$(A,B)$.  This approach leads to Algorithm~\ref{alg:gen_schur}.

If needed, eigenvectors can be computed in a manner that is similar to
any other generalized Schur decomposition.  Specifically, if
$\vec{u}_i$ is an eigenvector for $(I+\sigma T, T)$ corresponding to
eigenvalue $(1+\sigma\theta_i, \theta_i)$, then
\begin{equation*}
  \theta_i A Z \vec{u}_i 
  = \theta_i QT_1(I+\sigma T) \vec{u}_i
  = (1+\sigma \theta_i) Q T_1 T \vec{u}_i
  = (1+\sigma \theta_i) B Z \vec{u}_i
\end{equation*}
so that $\vec{v}_i = Z \vec{u}_i$ is an eigenvector for $(A,B)$.  

\begin{algorithm}
\caption{Shift and Invert Schur Decomposition}
\label{alg:gen_schur}
\begin{algorithmic}
\Function{GenSchur}{$A, B, \sigma$}
\State $\breve{A} \gets A - \sigma B$
\State Solve: $\breve{A}X = B$
\State Factor: $X = Z T Z^{\H}$
\State $W \gets \breve{A}Z$
\State Factor: $W = Q T_1$
\State \Return $(T, T_1, Q, Z)$
\EndFunction
\end{algorithmic}
\end{algorithm}

The following lemma is useful in the derivation of backward error
bounds for Algorithm~\ref{alg:gen_schur}.
\begin{lemma}
  \label{lm:pseudo_inv}
  For a given shift $\sigma$ with $\hat{\sigma}$ defined by
  \eqref{eq:scaled_eig_shift} and arbitrary matrix $T$, let
  \begin{equation*}
    C
    =
    \begin{bmatrix}
      -\|A\|_2 T / \|B\|_2 \\ I + \sigma T
    \end{bmatrix}.
  \end{equation*}
  Then the pseudoinverse $C^\dagger$ of $C$ satisfies
  \begin{equation*}
    \|C^{\dagger}\|_2\leq (1+|\hat{\sigma}|)
  \end{equation*}.
\end{lemma}
\begin{proof}
  We seek a lower bound on the smallest singular value of $C$.  We consider two
  cases.  Given a vector $\vec{x}$ with $\|\vec{x}\|_2 = 1$, either
  \begin{equation*}
    \|T\vec{x}\|_2 \leq \frac{\|B\|_2}{(1+|\hat{\sigma}|)\|A\|_2},
    \qquad\mbox{or}\qquad
    \|T\vec{x}\|_2 > \frac{\|B\|_2}{(1+|\hat{\sigma}|)\|A\|_2}.
  \end{equation*}
  In the first case we have
  \begin{equation*}
    \|(I + \sigma T) \vec{x}\|_2
    \geq \|\vec{x}\|_2 - |\sigma| \|T\vec{x}\|
    \geq 1 - \frac{|\sigma| \|B\|_2}{(1+|\hat{\sigma}|)\|A\|_2}
    = 1 - \frac{|\hat{\sigma}|}{(1+|\hat{\sigma}|)}
    = \frac{1}{1+|\hat{\sigma}|}.
  \end{equation*}
  The second case immediately gives
  \begin{equation*}
    \frac{\|A\|_2\|T\vec{x}\|_2}{\|B\|_2}  > \frac{1}{1+|\hat{\sigma}|}.
  \end{equation*}
  It follows that $\|C\vec{x}\| \geq 1/(1+|\hat{\sigma}|)$ for arbitrary
  $\vec{x}$ with $\|\vec{x}\|_2=1$ so that
  $\sigma_n(C)\geq 1/(1+|\hat{\sigma}|)$ and
  $\|C^{\dagger}\|_2\leq (1+|\hat{\sigma}|)$.
\end{proof}

\begin{theorem}
  \label{th:generalized_schur}
  Suppose that $A-\sigma B$ is nonsingular, and that there exist
  $a_n$, $b_n$, $c_n$, $d_n$, $e_n$, $f_n$, and $g_n$ depending solely
  on $n$ for which
  \begin{equation}
    \label{eq:shift_error}
    \|\breve{A} - (A-\sigma B)\|_2 \leq u a_n (\|A-\sigma B\|_2 + |\sigma| \|B\|_2),
  \end{equation}
  \begin{equation}
    \label{eq:Xerror}
    \left\| \breve{A} X - B \right\|_2 
    \leq u b_n \|\breve{A}\|_2\|X\|_2,
  \end{equation}
  \begin{equation}
    \label{eq:schur_error}
    \| X - \tilde{Z} T \tilde{Z}^\H \|_2 \leq u c_n \|X\|_2,\qquad
    \|Z - \tilde{Z}\|_2 \leq u d_n,
  \end{equation}
  \begin{equation}
    \label{eq:AZerror}
    \|\breve{A} Z - W\|_2 \leq u e_n \|\breve{A}\|_2,
  \end{equation}
  \begin{equation}
    \label{eq:QRerror}
    \|\tilde{Q}T_1 - W\|_2 \leq u f_n \|W\|_2, \eqand
    \|\tilde{Q} - Q\|_2 \leq ug_n ,
  \end{equation}
  where $\tilde{Q}$ and $\tilde{Z}$ are unitary, $T_1$ and $T$ are
  upper triangular, and $T$ is nonsingular.  Then there exist $E$ and
  $F$ satisfying
  \begin{equation*}
    \tilde{Q}^\H (A+E) \tilde{Z} = T_1(I+\sigma T),\eqand \tilde{Q}^\H (B+F) \tilde{Z}
    = T_1 T.
  \end{equation*}
  with
  \begin{alignat*}{3}
    \frac{\|E\|_2}{\|A\|_2} & \leq
    u \Big(&& |\hat{\sigma}| (b_n + c_n + d_n + e_n + f_n) K_2(A-\sigma B, B) \\
    & && + (1+|\hat{\sigma}|)(2a_n + d_n+e_n+f_n)\Big) +O(u^2)    
  \end{alignat*}
  and
  \begin{equation*}
    \frac{\|F\|_2}{\|B\|_2} \leq
    u \left( b_n + c_n +d_n +e_n + f_n\right) K_2(A-\sigma B, B) + O(u^2).
  \end{equation*}
\end{theorem}
\begin{proof}
  Let $\breve{A} X - B = G_1$, $X - \tilde{Z}T \tilde{Z}^\H = G_2$,
  $Z - \tilde{Z} = G_3$, $\breve{A}Z - W = G_4$, and
  $\tilde{Q}T_1 - W = G_5$.  The first two of these identities give
  \begin{equation*}
    \breve{A} \tilde{Z} T = B\tilde{Z} + G_1 \tilde{Z} - \breve{A}G_2 \tilde{Z}.
  \end{equation*}
  Continuing with various substitutions, we have
  \begin{equation*}
    \breve{A} Z T = B\tilde{Z} + G_1 \tilde{Z} - \breve{A}G_2 \tilde{Z} + \breve{A} G_3 T,
  \end{equation*}
  \begin{equation*}
    W T = B\tilde{Z} + G_1 \tilde{Z} - \breve{A}G_2 \tilde{Z} + \breve{A} G_3 T - G_4 T,
  \end{equation*}
  and
  \begin{equation*}
    \tilde{Q} T_1 T = B\tilde{Z} + G_1 \tilde{Z} - \breve{A}G_2 \tilde{Z} + \breve{A} G_3 T - G_4 T + G_5 T.
  \end{equation*}
  This implies that
  \begin{equation*}
    B + F = \tilde{Q} T_1 T \tilde{Z}^\H
  \end{equation*}
  where
  \begin{align*}
    \|F\|_2
    & \leq \|G_1\|_2 + \|\breve{A}\|_2 \|G_2\|_2 + \|\breve{A}\|_2 \|G_3\|_2 \|T\|_2 +
      \|G_4\|_2 \|T\|_2 + \|G_5\|_2 \|T\|_2 \\
    & = \begin{aligned}[t] & \|G_1\|_2 + \|\breve{A}\|_2 \|G_2\|_2 
                           + \|\breve{A}\|_2 \|G_3\|_2 \|X\|_2 +
                           \|G_4\|_2 \|X\|_2 + \|G_5\|_2 \|X\|_2   \\
                           & + O(u^2) \\
    \end{aligned} \\
    & \leq \begin{aligned}[t] u ( & b_n \|\breve{A}\|_2 \|X\|_2 
                                    + c_n \|\breve{A}\|_2 \|X\|_2 
                                    + d_n \|\breve{A}\|_2 \|X\|_2\\
                                  & \!+ e_n \|\breve{A}\|_2 \|X\|_2 
                                    + f_n \|W\|_2 \|X\|_2) +O(u^2)
      \end{aligned} \\
    & = u \left( b_n + c_n +d_n +e_n + f_n\right) \|A-\sigma B\|_2 \|X\|_2 + O(u^2) \\
    & = u \left( b_n + c_n +d_n +e_n + f_n\right) K_2(A-\sigma B, B) \|B\|_2 + O(u^2).
  \end{align*}
  
  Let $\breve{A} = A-\sigma B + G_0$.  Considering errors on $A$, we
  start with $W - G_5 = \tilde{Q}T_1$ and continue with substitutions
  \begin{equation*}
    \breve{A}Z - G_4 - G_5 = \tilde{Q}T_1,
  \end{equation*}
  \begin{equation*}
    (A-\sigma B + G_0)Z - G_4 - G_5 = \tilde{Q}T_1,
  \end{equation*}
  and
  \begin{equation*}
    (A-\sigma B + G_0)\tilde{Z} - G_4 + (A-\sigma B+G_0) G_3 - G_5  = \tilde{Q}T_1.
  \end{equation*}
  This implies that
  \begin{multline*}
    \tilde{Q}^\H \left(A + \sigma F + G_0 + \tilde{Q} \left[(A-\sigma B+G_0) G_3- G_4 -G_5\right] \tilde{Z}^\H\right)
    \tilde{Z} \\
    = T_1 + \sigma \tilde{Q}^\H(B+F) \tilde{Z} = T_1 (I+\sigma T),
  \end{multline*}
  or
  \begin{equation*}
    \tilde{Q}^\H (A+E) \tilde{Z} = T_1(I+\sigma T)
  \end{equation*}
  where
  \begin{align*}
    \|E\|_2 
    & \leq |\sigma| \|F\|_2 + \|G_0\|_2 + \|A-\sigma B\|_2 \|G_3\|_2 + \|G_4\|_2 + \|G_5\|_2 + O(u^2) \\
    & \leq 
      \begin{aligned}[t]
        & u |\sigma| (b_n + c_n + d_n + e_n + f_n) K_2(A-\sigma B, B) \|B\|_2 
          + ua_n |\sigma| \|B\|_2 \\
        & \!+ u(a_n + d_n+e_n+f_n) \|A-\sigma B\|_2 +O(u^2)\\
      \end{aligned} \\
    & \leq 
      \begin{aligned}[t] 
        & u |\hat{\sigma}| (b_n + c_n + d_n + e_n + f_n) 
          K_2(A-\sigma B, B) \|A\|_2 + ua_n |\hat{\sigma}| \|A\|_2 \\
        & \! + u(a_n + d_n+e_n+f_n) (\|A\|_2 + |\hat{\sigma}| \|A\|_2) +O(u^2)\\
      \end{aligned} \\
    & \leq 
      \begin{aligned}[t] 
        u \Big(& |\hat{\sigma}| (b_n + c_n + d_n + e_n + f_n) K_2(A-\sigma B, B) \\
        & \! + (1+|\hat{\sigma}|)(2a_n + d_n+e_n+f_n)\Big) \|A\|_2 +O(u^2).\\
      \end{aligned} \\
  \end{align*}
\end{proof}

Algorithm~\ref{alg:gen_schur} computes a generalized Schur decomposition from
$B$ and $A-\sigma B$.  Recalling \eqref{eq:gamma_def}, we note that
$1/\gamma = \|A\|_2 / \|A-\sigma B\|_2$ is large only if there is extreme
cancellation in forming $A-\sigma B$.  Cancellation did not turn out to be a
concern in Theorem~\ref{th:generalized_schur}, but the quantity
$|\hat{\sigma}|/\gamma$ will appear in the residual bounds.  This quantity is
large only if $1/\gamma$ is large, regardless of the magnitude of
$|\hat{\sigma}|$.  To see this note that
\begin{equation*}
  \frac{|\hat{\sigma}|}{\gamma}
  = \frac{\hat{\sigma}}{\|\hat{A} - \hat{\sigma} \hat{B}\|_2}
  \leq \frac{|\hat{\sigma}|}{| \|\hat{A}\|_2 - |\hat{\sigma}| \|\hat{B}\|_2|}
  = \frac{|\hat{\sigma}|}{| 1 - |\hat{\sigma}| |}.
\end{equation*}
If $|\hat{\sigma}| \leq 2$, then $|\hat{\sigma}/\gamma| \leq 2/\gamma$.  If
$|\hat{\sigma}| > 2$, then the above inequality implies that
$|\hat{\sigma}| / \gamma \leq 2$.  Thus, with no assumption on $\hat{\sigma}$,
we have
\begin{equation*}
  \label{eq:sigma0_gamma_bound}
  \frac{|\hat{\sigma}|}{\gamma} \leq 2\max\left(\frac{1}{\gamma}, 1\right).
\end{equation*}


\begin{theorem}
  \label{th:residual_bounds}
  Assume that \eqref{eq:shift_error} and \eqref{eq:Xerror} hold.
  Suppose that a computed eigenvalue $\theta\neq 0$ and eigenvector
  $\vec{v}$ of $X$ satisfy
  \begin{equation*}
    X\vec{v} - \theta \vec{v} = \vec{r},
    \qquad\text{with}\qquad
    \|\vec{r}\|_2 \leq u g_n \|X\|_2 \|\vec{v}\|_2.
  \end{equation*}
  Let $\lambda = \sigma + 1/\theta$, with $\lambda = \infty$ if
  $\theta = 0$.  Then
  \begin{alignat}{3}
    \label{eq:eq:res_bound_large_eig}
    \| \theta A \vec{v} - (1+\sigma \theta) B\vec{v} \|_2 
    & \leq && u |1+\sigma \theta| \cdot |1-\sigma/\lambda|
    \Big( b_n + g_n  + a_n (1+2\max(1/\gamma, 1))  \Big) \nonumber\\
    & && \cdot K_2(A-\sigma B, B)\|B\|_2 \|\vec{v}\|_2 + O(u^2),
  \end{alignat}
  where we take $|1-\sigma/\lambda| = 1$ if $\lambda = \infty$ and
  $|1-\sigma/\lambda| = \infty$ if $\lambda = 0$ so that the bound is
  vacuous if $\lambda = 0$.  We also have
  \begin{alignat}{3}
    \label{eq:eq:res_bound_small_eig}
    \| \theta A \vec{v} - (1+\sigma \theta) B\vec{v} \|_2
    & \leq && u |\theta| \cdot |\hat{\sigma}| \cdot |1-\lambda/\sigma|
    \Big( b_n + g_n  + a_n (1+2\max(1/\gamma, 1))  \Big) \nonumber\\
    & && \cdot K_2(A-\sigma B, B) \|A\|_2 \|\vec{v}\|_2 
    + O(u^2).
  \end{alignat}
\end{theorem}
\begin{proof}
  With $G_0$ and $G_1$ as defined in the proof of Theorem~\ref{th:generalized_schur} we have
  \begin{equation*}
    (\breve{A}-G_0)X \vec{v} - \theta(A-\sigma B) \vec{v} = (A-\sigma B)\vec{r},
  \end{equation*}
  \begin{equation*}
    (B + G_1 - G_0 X)\vec{v}- \theta(A-\sigma B) \vec{v} = (A-\sigma B)\vec{r},
  \end{equation*}
  and
  \begin{equation*}
    \theta A \vec{v} - (1+\sigma \theta) B\vec{v} = G_1\vec{v} - G_0 X \vec{v} - (A-\sigma B)\vec{r}.
  \end{equation*}
  It follows that
  \begin{align}
    \| \theta A \vec{v} - (1+\sigma \theta) B\vec{v} \|_2 
    & \leq 
      \begin{aligned}[t] 
        u \Big( & b_n \|\breve{A}\|_2\|X\|_2 + a_n (\|\breve{A}\|_2
          + |\sigma| \|B\|_2) \|X\|_2 \\
        & + g_n \|A-\sigma B\|_2 \|X\|_2 \Big) \|\vec{v}\|_2
      \end{aligned} \nonumber\\
    & = 
      \begin{aligned}[t]
        u \Big( & (a_n+b_n+g_n) K_2(A-\sigma B, B) \|B\|_2 \\
                & + a_n |\hat{\sigma}| \|A\|_2 \|X\|_2\Big) \|\vec{v}\|_2 + O(u^2) 
      \end{aligned} \nonumber\\
    & = u \Big( b_n + g_n  + a_n (1+|\hat{\sigma}| / \gamma)  \Big) 
      K_2(A-\sigma B, B) \|B\|_2
      \|\vec{v}\|_2 \label{eq:res_proof_bound}.
  \end{align}
  We note that
  \begin{equation*}
    1+\sigma \theta 
    = 1 + \sigma \frac{1}{\lambda - \sigma} 
    = \frac{1}{1-\sigma/\lambda},
  \end{equation*}
  which together with \eqref{eq:res_proof_bound} and
  \eqref{eq:sigma0_gamma_bound} implies \eqref{eq:eq:res_bound_large_eig}.  With
  \begin{equation*}
    \theta 
    = \frac{1}{\lambda - \sigma} = \frac{\|B\|_2}{\hat{\sigma} \|A\|_2 (\lambda/\sigma - 1)}
  \end{equation*}
  we obtain
  \begin{equation*}
    \|B\|_2 = |\hat{\sigma}(1- \lambda/\sigma)| \|A\|_2,
  \end{equation*}
  which substituted into \eqref{eq:res_proof_bound} gives
  \eqref{eq:eq:res_bound_small_eig}.
\end{proof}

\section{Residual Bounds}
\label{sec:residual-bounds}



\bibliography{/home/mas/work/bib/ref}


\end{document}
