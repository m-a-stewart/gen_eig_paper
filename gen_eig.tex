\documentclass{siamltex}
\usepackage[bookmarks=true]{hyperref}
% \usepackage{refcheck}
% \usepackage[margin=1.25in]{geometry}
\usepackage{mathtools}
\usepackage{amsmath}
% \usepackage{amsthm}
\usepackage{amsfonts}
\usepackage{bm}
\usepackage{amssymb}
\usepackage{bbold}
\usepackage{algorithm}
\usepackage[noEnd=false,indLines=false]{algpseudocodex}
\usepackage{colortbl}
\usepackage{multirow}
\usepackage{url}
\usepackage{placeins}
\usepackage{hhline}
% \usepackage{colonequals}
% \usepackage{tikz}
% \usepackage{pgfplots}
% \usepgfplotslibrary{fillbetween}
% \usepackage{graphicx}
\DeclareMathOperator*{\argmin}{arg\,min}
% \DeclareMathOperator{\diag}{diag}
\DeclareMathOperator{\real}{Re}
\DeclareMathOperator{\imag}{Imag}
\DeclareMathOperator{\fl}{fl}
\DeclareMathOperator{\sign}{sign}
% \DeclareMathOperator{\rank}{rank}
\DeclareMathOperator{\trace}{Tr}
\newcommand{\R}{\mathbb{R}}\newcommand{\C}{\mathbb{C}}\newcommand{\Z}{\mathbb{Z}}
\newcommand{\qed}{\rule{1.2ex}{1.2ex}}
\newcommand{\eqand}{\qquad\text{and}\qquad}
\newcommand{\eqor}{\qquad\text{or}\qquad}
\newcommand{\conj}[1]{\overline{#1}}
\renewcommand{\vec}[1]{\boldsymbol{#1}}
\newcommand{\T}{\mathrm{T}}
\renewcommand{\H}{\mathrm{H}}
\newcommand{\mcal}{\mathcal}
% \newtheorem{theorem}{Theorem}
% \newtheorem{lemma}{Lemma}
% \newtheorem{corollary}{Corollary}
\newtheorem{example}{Example}

% Add optional column specifiers for matrix environments.
\makeatletter
\renewcommand*\env@matrix[1][*\c@MaxMatrixCols c]{%
  \hskip -\arraycolsep
  \let\@ifnextchar\new@ifnextchar
  \array{#1}}
\makeatother

% Put paragraph indentation in enumerate environment.
% \let\oldenumerate=\enumerate
% \renewenvironment{enumerate}{\oldenumerate\parindent=1.5em}{\endlist}

%  Dark mode: Turn off PDF mode by C-c C-t C-p and set `preview-image-type' to
%  `dvipng'.
% (setq preview-image-type 'dvipng)

\bibliographystyle{siam}
 \title{Pseudospectra and the Shift and Invert Solution
  of the Generalized Eigenvalue Problem}
\author{Michael Stewart\thanks{Department of Mathematics and
    Statistics, Georgia State University, Atlanta GA 30303, {\tt
      mastewart@gsu.edu}}}
\pagestyle{myheadings}
\thispagestyle{plain}
\markboth{MICHAEL STEWART}{SHIFT AND INVERT SOLUTION
  OF THE GENERALIZED EIGENVALUE PROBLEM}

\begin{document}
\maketitle
\begin{abstract}
  Given matrices $A$ and $B$, this paper gives error bounds for the solution of
  the generalized eigenvalue problem using a shift and invert strategy.  The
  analysis identifies circumstances under which the solution can be expected to
  satisfy useful error bounds.  The conditions are notably less restrictive than
  requiring that a shifted be well conditioned.
\end{abstract}
\begin{keywords}
  generalized eigenvalues, eigenvalues, eigenvectors, error analysis
\end{keywords}
\begin{AMS}
  15A18, 15A22, 15A23, 15A42, 65F15
\end{AMS}

\section{Background}
\label{sec:background}

In what follows we assume that $A$ and $B$ are $n\times n$ nonzero complex
matrices that are not necessarily hermitian.  Our goal is to solve the
generalized eigenvalue problem
\begin{equation}
  \label{eq:gen_eig1}
  A\vec{v}_i = \lambda B \vec{v}_i, \qquad \vec{v}_i \neq \vec{0}, \qquad
  i=1,2,\ldots n.
\end{equation}
We also use the formulation
\begin{equation}
  \label{eq:gen_eig2}
  \beta_i A \vec{v}_i = \alpha_i B \vec{v}_i, \qquad \vec{v}_i\neq \vec{0}, \qquad
  i = 1,2,\ldots, n.
\end{equation}
where $\beta_i$ and $\alpha_i$ are not both zero and
$\lambda_i = \alpha_i/\beta_i$.  If $\beta_i = 0$, we let $\lambda_i = \infty$.
If $A$ and $B$ have a common null space, then any common null vector would
satisfy \eqref{eq:gen_eig1} or \eqref{eq:gen_eig2} for any choice of $\lambda$
or $\alpha$ and $\beta$.  We therefore require that the intersection of the
null spaces of $A$ and $B$ be trivial, that is that the pencil $A-\lambda B$ be
\textit{regular}.  The problem is fully defined by the pair $(A,B)$ and the
generalized eigenvalues by $(\alpha_i, \beta_i)$.  When the context is clear,
we will often simply refer to eigenvalues and eigenvectors instead of
generalized eigenvalues and eigenvectors.  We also refer to both $\lambda_i$
and the pair $(\alpha_i, \beta_i)$ as eigenvalues.

If $B$ is nonsingular, then \eqref{eq:gen_eig1} can be converted to the
ordinary eigenvalue problem by solving the ordinary eigenvalue problem
$B^{-1}A$, which in practice would be computed using some stable factorization
of $B$.  If $A$ is nonsingular, then each eigenvalue of $A^{-1}B$ is of the
form $1/\lambda_i$, where $\lambda_i$ satisfies (\ref{eq:gen_eig1}).  Either
formulation can be used to solve (\ref{eq:gen_eig1}).  However, if the matrix
to be inverted is ill conditioned then the computed $B^{-1}A$ (or $A^{-1}B$)
might have large errors.  In this case, the computed generalized eigenvalues
might be inaccurate, even when they are well conditioned \cite{govl:13,
  stew:01}.

To try to avoid problems with ill conditioning of $A$ when forming $A^{-1}B$,
we can choose a shift for which $A-\sigma B$ is nonsingular and instead form
\begin{equation}
  \label{eq:X-def}
  X = (A-\sigma B)^{-1} B.
\end{equation}
If
\begin{equation}
  \label{eq:X_problem}
  X \vec{v}_i = \theta_i \vec{v}_i, \qquad \vec{v}_i \neq 0, 
  \qquad i=1,2,\ldots, n,
\end{equation}
then (\ref{eq:gen_eig2}) holds with
\begin{equation*}
  \alpha_i = 1+\sigma \theta_i, \eqand \beta_i = \theta_i.
\end{equation*}
For each $\theta_i\neq 0$ we then have a finite generalized eigenvalue
\begin{equation*}
  \lambda_i = (1+\sigma \theta_i)/\theta_i = \frac{1}{\theta_i} + \sigma
\end{equation*}
in (\ref{eq:gen_eig1}).  If $\theta_i = 0$, then $\vec{v}_i$ is an eigenvector
for $\lambda_i=\infty$.  Note that either $\alpha_i$ or $\beta_i$ must be
nonzero.  This paper focuses on the numerical properties of the eigenvalue
problem (\ref{eq:X_problem}) as a means for solving (\ref{eq:gen_eig2}).

Unfortunately, it is not always possible to choose $\sigma$ so that
$A-\sigma B$ is well conditioned.  Even with the use of a shift, the explicit
conversion of the generalized eigenvalue problem to an ordinary eigenvalue
problem is not commonly recommended, except perhaps in the use of iterative
methods like the Arnoldi algorithm.  The $QZ$ algorithm \cite{most:73} instead
solves the generalized eigenvalue by computing orthogonal $Q$ and $Z$ for which
\begin{equation*}
  A = Q T_a Z^\H, \eqand B = QT_b Z^\H
\end{equation*}
where $T_a$ and $T_b$ are upper triangular.  If the $i$th diagonal elements of
$T_a$ and $T_b$ are $\alpha_i$ and $\beta_i$ respectively, then each
$(\alpha_i, \beta_i)$ is an eigenvalue of \eqref{eq:gen_eig2}.  The
decomposition is perfectly backward stable so that for $i=1,\ldots, n$ the
pairs $(\alpha_i, \beta_i)$ form the set of eigenvalues of a pair of matrices
close to $(A,B)$.  The generalized eigenvectors can be obtained from $Z$ in a
manner analogous to what is done with a Schur decomposition for the ordinary
eigenvalue problem.  The only disadvantage of this method for a general dense
problem is the computational cost of the $QZ$ algorithm relative to that of
solving an ordinary eigenvalue problem for $(A-\sigma B)^{-1}B$.  However,
given the possible instability associated with forming $(A-\sigma B)^{-1}B$,
the $QZ$ algorithm is the standard method for the dense nonhermitian
generalized eigenvalue problem.

In this paper, we revisit the use of inversion to convert the generalized
eigenvalue problem to an ordinary eigenvalue problem and show that, while not
backward stable, it can be more stable than consideration of the condition
number of $A-\sigma B$ might suggest.  Further, the instability is detectable
prior to computing an eigenvalue decomposition of $X$ and can be quantified in
backward error and residual bounds.  In particular, if we define
\begin{equation}
  \label{eq:K2_C_def}
  K_2(C, B) = \frac{\|C\|_2 \|C^{-1} B\|_2}{\|B\|_2}
\end{equation}
then we will see in the error bounds that
\begin{equation}
  \label{eq:K2_def}
  K_2(A-\sigma B, B) = \frac{\|A-\sigma B\|_2}{\|B\|_2} \|(A-\sigma B)^{-1} B\|_2
\end{equation}
is the key factor affecting the stability, or lack of stability, of the shift
and invert approach.  Note that if $B$ is nonsingular then
$\|(A-\sigma B)^{-1}B\|_2= \|(B^{-1}A-\sigma I)^{-1}\|_2$ is the resolvent norm
of $B^{-1}A$, which immediately suggests a connection to the pseudospectrum of
$B^{-1}A$.

If we define a scaled shift and scaled eigenvalues
\begin{equation}
  \label{eq:scaled_eig_shift}
  \hat{\sigma} = \frac{\|B\|_2}{\|A\|_2} \sigma, \qquad
  \hat{\lambda}_i = \frac{\|B\|_2}{\|A\|_2} \lambda_i, \qquad
  i = 1, 2,\ldots, n,
\end{equation}
and
\begin{equation*}
  \hat{A} = \frac{A}{\|A\|_2}, \eqand
  \hat{B} = \frac{B}{\|B\|_2},
\end{equation*}
then the condition number of $A-\sigma B$ satisfies $\kappa_2(A-\sigma B) = \kappa_2(\hat{A} - \hat{\sigma} \hat{B})$.
For $K_2(A-\sigma B, B)$, we have the similar identity
\begin{equation*}
  K_2(A-\sigma B, B)
  = \|\hat{A} - \hat{\sigma} \hat{B}\|_2
  \|(\hat{A}-\hat{\sigma} \hat{B})^{-1} \hat{B}\|_2
  = K_2(\hat{A}-\hat{\sigma} \hat{B}, \hat{B}).
\end{equation*}
Thus $K_2(A- \sigma B, B)$ does not change with scaling of $A$ and $B$, so long as
the shift is scaled accordingly.

It is shown in section~\ref{sec:pseud-bounds-k2} that
\begin{equation}
  \label{eq:upper_lower_bounds}
  \frac{\|\hat{A}-\hat{\sigma} \hat{B}\|_2}{\min_i |\hat{\lambda_i} - \hat{\sigma}|}
  \leq K_2(A-\sigma B,B) 
  \leq \kappa_2(\hat{A}-\hat{\sigma} \hat{B}),
\end{equation}
where the minimum is over all scaled generalized eigenvalues $\hat{\lambda}_i$
of the pair $(A,B)$, assuming $1/|\hat{\lambda}_i - \hat{\sigma}| = 0$ if
$\lambda_i = \infty$.  The upper bound is tight when
$\|(A-\sigma B)^{-1} B\|_2 = \|(A-\sigma B)^{-1} \|_2 \|B\|_2$.  It is perhaps worth noting
that this always holds when $B=I$, so that
\begin{equation*}
  K_2(A-\sigma I, I) = \kappa_2(\hat{A} - \hat{\sigma} I) 
  = \kappa_2(A - \sigma I).
\end{equation*}
The lower bound will be seen to be an equality if $(A,B)$ has a complete
orthonormal set generalized eigenvectors.  If $B$ is nonsingular, this is
equivalent to $B^{-1}A$ being normal.  In the case in which $(A,B)$ has
orthonormal eigenvectors, it is often not too difficult to select a shift for
which $K_2(A - \sigma B, B)$ is not large.


The structure of the paper is as follows.  In section~\ref{sec:pseudospectra} we
consider a definition of the pseudospectrum of the pair $(A,B)$.  In
section~\ref{sec:pseud-bounds-k2} we give a characterization of $K_2(A-\sigma B,B)$ in
terms of the pseudospectrum of $(A,B)$.  In section~\ref{sec:schur-decomp} we
give error bounds for an algorithm that uses shift and invert to compute a
generalized Schur decomposition of $(A,B)$.  In section~\ref{sec:residual-bounds} we
present bounds that relate residuals for the ordinary eigenvalue problem
(\ref{eq:X_problem}) to residuals for the generalized eigenvalue problem
(\ref{eq:gen_eig2}).  The magnitude of $K_2(A-\sigma B, B)$ is the key measure
of instability in both section~\ref{sec:schur-decomp} and
section~\ref{sec:residual-bounds}.

\section{Pseudospectra of $(A,B)$}
\label{sec:pseudospectra}

In this section, we consider the choice of shift and its impact on
$K_2(A-\sigma B, B)$.  A rough summary of where we will end up is that we can
expect to have useful error bounds when $\sigma$ is not in the
$\epsilon$-pseudospectrum of $B^{-1}A$ for small $\epsilon$.  However, we do
not want to assume that $B$ is nonsingular.  Typically one would declare that
$\sigma$ is in the $\epsilon$-pseudospectrum of $B^{-1}A$ if there exists $E$
with $\|E\|_2 < \epsilon$ for which $\sigma$ is an eigenvalue of
$B^{-1} A - E$.  Instead, we say that $\sigma$ is in the
$\epsilon$-pseudospectrum of $(A,B)$ if there exists $E$ with
$\|E\|_2 < \epsilon$ for which $\sigma$ is a generalized eigenvalue of
$(A - BE, B)$.  The definitions are equivalent when $B$ is invertible but
latter does not require invertibility.  We expand on this idea and will show
that other known properties of pseudospectra can be formulated so as to avoid
inversion of $B$.

This approach to defining the pseudospectrum for generalized eigenvalue
problems is very straight-forward, but seems to be distinct from what we have
seen elsewhere.  In \cite{hiti:02}, both $A$ and $B$ are perturbed, which means
ythat when $B$ is nonsingular, the results do not match the pseudospectrum of
$B^{-1}A$.  In \cite{emke:17}, a Schur factorization is used to isolate zero
eigenvalues of $(A-\sigma B)^{-1}B$, corresponding to infinite eigenvalues of
$(A,B)$ in a nilpotent block in order to remove them from consideration.
Several other approaches are described in \cite{trem:05}, none of which is
equivalent to what we consider.

Let $\widehat{\C} = \C\cup \{\infty\}$ and let
$\Lambda_0(A,B)\subseteq \widehat{\C}$ denote the set of generalized
eigenvalues of $(A,B)$.  We assume throughout that $(A,B)$ is a regular pair.
We adopt several conventions for dealing with expressions involving $\infty$.
In particular, along with $1/\infty = 0$ and $1/0=\infty$, we assume that
\begin{equation*}
\text{$\|(A-\sigma B)^{-1}B\|_2 = \infty$ for
$\sigma\in\Lambda_0(A,B)$},
\end{equation*}
and
\begin{equation*}
\text{$\|(A-\infty B)^{-1}B\|_2 = 0$ for $\infty \notin \Lambda_0(A,B)$.}
\end{equation*}

We will use the fact that the pair $(A-\sigma B,B)$ has a generalized singular
value decomposition
\begin{equation}
  \label{eq:gen-svd}
  A -\sigma B = S \Sigma_a U^\H, \eqand
  B = S \Sigma_b V^\H,
\end{equation}
where $\Sigma_a$ and $\Sigma_b$ are diagonal, $S$ is nonsingular, and $U$ and
$V$ are unitary, \cite{pasa:81, vloa:76}.  If $\sigma \notin \Lambda_0(A,B)$ so
that $A-\sigma B$ is nonsingular, then the columns of $S$ can be scaled so that
$\Sigma_a = I$.  If $B$ is invertible, then the singular values of
$B^{-1} (A-\sigma B)$ are the diagonal elements $s_i$ for $i=1,2,\ldots, n$ of
$\Sigma_b^{-1}\Sigma_a$.  These are the generalized singular values of
$(A-\sigma B, B)$.  If $B$ has rank $r<n$ with
$\Sigma_b = \Sigma_{b,1}\oplus 0_{n-r\times n-r}$ and
$\Sigma_{a} = \Sigma_{a,1} \oplus \Sigma_{a, 2}$ where $\Sigma_{a, 1}$ is
$r\times r$, then the finite generalized singular values of $(A-\sigma B, B)$
are the diagonal elements $s_i$ of $\Sigma_{b,1}^{-1} \Sigma_{a,1}$ for
$i=1,2,\ldots, r$ and there are $n-r$ infinite generalized singular values
$s_i=\infty$ for $i=r+1, \ldots, n$ corresponding to the zero diagonal elements
of $\Sigma_b$.

For $0 < \epsilon < \infty$, we define the $\epsilon$-pseudospectrum of $(A,B)$
by
\begin{equation}
  \label{eq:pseudo_def}
  \Lambda_\epsilon(A,B) = \left\{ \sigma \in \widehat{\C} :
    \text{$\sigma \in \Lambda_0(A-BE, B)$ for some $E$
      with $\|E\|_2 < \epsilon$} \right\}.
\end{equation}
When $B$ happens to be nonsingular, this is easily seen to be the
$\epsilon$-pseudospectrum of $B^{-1}A$ \cite{trem:05}.  The pair $(A-BE, B)$
has infinite generalized eigenvalues if and only if $B$ is singular and, hence,
if and only if $(A, B)$ has infinite eigenvalues.  It follows that for any
$\epsilon > 0$, $\infty \in \Lambda_\epsilon(A,B)$ if and only if
$\infty \in \Lambda_0(A,B)$.  If we define
\begin{equation*}
  \mcal{E}(A,B,\sigma) = \{E : \sigma \in \Lambda_0(A- BE, B)\},
\end{equation*}
then we can restate \eqref{eq:pseudo_def} as
\begin{equation*}
  \Lambda_\epsilon(A,B) = \left\{ \sigma \in \widehat{\C} :
    \text{there exists $E\in \mcal{E}(A,B,\sigma)$ with $\|E\|_2 < \epsilon$} \right\}.
\end{equation*}
It is not difficult to see that $\mcal{E}(A,B,\sigma)$ is closed for all
$\sigma \in \widehat{C}$.  The set $\mcal{E}(A,B,\sigma)$ has the inconvenient
property that it may be empty.  The analogous set for the ordinary
$\epsilon$-pseudospectrum of $A$ is $\{E : \text{$\sigma$ is an eigenvalue of
  $A-E$}\}$ and it is never empty.

Let
\begin{equation}
  \label{eq:delta_def}
  \delta(A, B, \sigma) 
  = \inf_{E\in \mcal{E}(A,B,\sigma)} \|E\|_2,
\end{equation}
with the convention that if the set $\mcal{E}(A,B,\sigma)$ is empty, then
$\delta(A, B, \sigma) = \infty$.  It is clear that if $\mcal{E}(A,B,\sigma)$ is
nonempty then $\delta(A, B, \sigma) < \infty$ so that $\mcal{E}(A,B,\sigma)$ is
empty if and only if $\delta(A,B,\sigma)=\infty$.  It is also clear that for
$\sigma \in \widehat{\C}$, if $\sigma \in \Lambda_0(A,B)$, then
$\delta(A,B,\sigma)=0$.  We refer to $\delta(A,B,\sigma)$ as the
\textit{pseudospectral separation} of $\sigma$ from $\Lambda_0(A,B)$.  The
definition of $\delta(A,B,\sigma)$ gives another way to restate \eqref{eq:pseudo_def}:
\begin{equation}
  \label{eq:pseudo-def-delta}
  \Lambda_\epsilon(A,B) = \left\{ \sigma \in \widehat{\C} 
    : \delta(A,B,\sigma) < \epsilon\right\}.
\end{equation}
To see this, we note that if $\sigma\in \Lambda_{\epsilon}(A,B)$, then there
exists $E$ with $\|E\|_2 < \epsilon$ and $E \in \mcal{E}(A,B,\sigma)$ so that
$\delta(A,B,\sigma) \leq \|E\|_2 < \epsilon$.  Conversely, if
$\delta(A,B,\sigma) = \epsilon_0 < \epsilon$, then $\mcal{E}(A,B,\sigma)$ is
nonempty and there exists $E_0\in\mcal{E}(A,B,\sigma)$ with
$\|E_0\|_2 < \epsilon$ so that $\sigma \in \Lambda_\epsilon(A,B)$.

The cases in which $\mcal{E}(A,B,\sigma)$ is empty (equivalently
$\delta(A,B,\sigma)=\infty$) are characterized in the following theorem.  The
cases are precisely those in which no choice of $E$ will make $\sigma$ a
generalized eigenvalue of $(A-BE, B)$.
\begin{theorem}
  \label{th:E-empty}
  For regular $(A,B)$, the set $\mcal{E}(A,B,\sigma)$ is empty if and only if
  one of the following conditions holds
  \begin{equation*}
    \text{$\sigma=\infty$ and $B$ is nonsingular}, \eqor
    \text{$\sigma=\C$ and $B = 0$}.
  \end{equation*}
\end{theorem}
\begin{proof}
  The first case implies that $\mcal{E}(A,B,\sigma)$ is empty because there is
  no way that $(A-BE,B)$ can have an infinite generalized eigenvalue when $B$
  is nonsingular.  The second case implies $\mcal{E}(A,B,\sigma)$ is empty
  because if $B=0$ regularity of $(A,B)$ implies that $A$ must be nonsingular
  so that there is no way that $(A-BE, B) = (A,0)$ can have a finite
  generalized eigenvalue satisfying $A\vec{x} = \lambda B\vec{x} = \vec{0}$.

  The negation of the two conditions of the theorem corresponds to three cases:
  \begin{equation*}
    \text{$\sigma = \infty$ and $B$ is singular}; \quad
    \text{$B$ is singular and $B\neq 0$}; \quad\text{or}\quad
    \text{$\sigma \in \C$ and $B\neq 0$}.
  \end{equation*}
  The middle case implies one of the two others depending on whether
  $\sigma\in\C$ or $\sigma=\infty$, so that we need to consider only the first
  and the last cases.  We show that if either of these hold, then
  $\mcal{E}(A,B,\sigma)$ is nonempty.  The first condition is trivial, since if
  $B$ is singular, then $\sigma = \infty$ is a generalized eigenvalue of
  $(A-BE, B)$ for any $E$.  For the third condition, if
  $\sigma \in \Lambda_0(A,B)$, then $E=0$ is in $\mcal{E}(A,B,\sigma)$.  If
  $\sigma \notin \Lambda_0(A,B)$, then let $(A-\sigma B, B)$ have generalized
  SVD \eqref{eq:gen-svd} with $\Sigma_a = I$.  We have
  \begin{equation*}
    A - BE - \sigma B = S (I -\Sigma_b V^\T E U) U^\T.
  \end{equation*}
  Suppose without loss of generality that $(1,1)$ element of $\Sigma_b$ is
  the largest generalized singular value $s_1\neq 0$ of $(A-\sigma B, B)$.  Let
  \begin{equation}
    E = \frac{1}{s_1} V \vec{e}_1 \vec{e}_1^\T U^\T.
  \end{equation}
  then
  \begin{equation*}
     I-\Sigma_b V^\T E U = I - \frac{1}{s_1} \Sigma_b \vec{e}_1 \vec{e}_1^\T
  \end{equation*}
  has $\vec{e}_1$ as a null vector so that $A-BE - \sigma B$ is singular and
  $E \in \mcal{E}(A,B,\sigma)$.
\end{proof}

If $B$ is nonsingular and $\sigma$ is not an eigenvalue of $B^{-1}A$, then it
is well known that $\sigma$ is in the $\epsilon$-pseudospectrum of $B^{-1}A$ if
and only if $\|(B^{-1}A - \sigma I)^{-1}\|_2 > 1/\epsilon$.  The following
lemma reformulates this result to avoid assuming invertibility of $B$.
\begin{theorem}
  \label{th:resolvent_bound}
  Let $(A,B)$ be regular.  With $\delta(A, B, \sigma)$ defined as in
  \eqref{eq:delta_def}, we have
  \begin{equation*}
    \|(A-\sigma B)^{-1}B\|_2 = \frac{1}{\delta(A,B,\sigma)}.
  \end{equation*}
\end{theorem}
\begin{proof}
  If $\sigma \in \Lambda_0(A,B)$ then $1/\delta(A,B,\sigma) = \infty$ and,
  by convention, $\|(A-\sigma B)^{-1}B\|_2 = \infty$.  So the theorem holds
  in this case and we now assume that $\sigma \notin\Lambda_0(A,B)$.

  If $\mcal{E}(A,B,\sigma)$ is empty so that $1/\delta(A,B,\sigma) =0$, then
  one of the two conditions in Theorem~\ref{th:E-empty} hold.  If $\sigma\in\C$
  and $B=0$, then $\|(A-\sigma B)^{-1}B\|_2 = 0$.  If $\sigma=\infty$ and $B$
  is nonsingular, then by convention $\|(A-\infty B)^{-1}B\|_2 = 0$.

  We assume now that $\mcal{E}(A,B,\sigma)$ is nonempty and
  $\sigma \notin \Lambda_0(A,B)$.  Considering the three cases obtained from
  the negation of the conditions of Theorem~\ref{th:E-empty}, the middle case
  is redundant and $\sigma \notin \Lambda_0(A,B)$ rules out the first case, so
  we can assume the third case, that $\sigma\in\C$ and $B\neq 0$.  The matrix
  $A-\sigma B$ is nonsingular and we consider the generalized singular value
  decomposition \eqref{eq:gen-svd} with $\Sigma_a = I$.  We have
  $(A-\sigma B)^{-1} B = U \Sigma_b V^\H$, and without loss of generality, we
  can assume that $\Sigma_b = \diag(b_1, \ldots, b_n)$, $b_1 > 0$, and
  $b_k\geq 0$ is nonincreasing for $k=1,2,\ldots, n$.  We recall that the
  diagonal elements $s_k$ of $\Sigma_b^{-1} \Sigma_a$ are the generalized
  singular values of $(A-\sigma B, B)$, where zeros on the diagonal of
  $\Sigma_b$ correspond to infinite generalized singular values.  We then see
  that $s_1 = 1/b_1$ is the smallest generalized singular value of
  $(A-\sigma B)^{-1}B$.  Clearly
  $A - BE -\sigma B = S (I - \Sigma_b V^\H E U) U^\H$ is singular if and only
  if $(I - \Sigma_b V^\H E U)$ is singular which implies that
  $\|E\|_2 \geq 1/\|\Sigma_b\|_2$.  Thus $\|E\|_2 \geq 1/b_1 = s_1$ for all
  $E\in\mcal{E}(A,B,\sigma)$ and
  \begin{equation*}
    \delta(A,B,\sigma)
    = \inf_{E \in \mcal{E}(A,B,\sigma)} \|E\|_2
    \geq s_1
    = \frac{1}{\|(A-\sigma B)^{-1}B\|_2}.
  \end{equation*}
  As in the proof of Theorem~\ref{th:E-empty}, we can set
  \begin{equation*}
    E = s_1 V \vec{e}_1 \vec{e}_1^\H U^\H
  \end{equation*}
  so that $A-BE - \sigma B = S (I-\vec{e}_1 \vec{e}_1^\H) U^\H$ is singular
  and 
  \begin{equation*}
    \delta(A,B,\sigma)
    \leq \|E\|_2 
    = s_1 = \frac{1}{\|(A-\sigma B)^{-1} B\|_2}.
  \end{equation*}
\end{proof}

For the ordinary pseudospectrum, $\sigma\in \C$ is in the $\epsilon$-pseudospectrum
of $A$ when $s_{\min}(A-\sigma I) < \epsilon$, where $s_{\min}$ denotes the minimum
singular value.  The statement and proof of Theorem~\ref{th:resolvent_bound}
shows that we have $\sigma\in\Lambda_\epsilon(A,B)$ when the smallest
generalized singular value of $(A-\sigma B, B)$ is less than $\epsilon$.

Another definition of the ordinary pseudospectrum is that $\sigma$ is in the
$\epsilon$-pseudospectrum of $A$ when there exists some vector $\vec{v}$ with
$\|\vec{v}\|_2=1$ and $\|(A - \sigma I)\vec{v}\|_2 < \epsilon$.  A formulation
of this for the generalized pseudospectrum is that
$\sigma\in\Lambda_\epsilon(A,B)$ when there exist $\vec{v}$ with
$\|\vec{v}\|_2 = 1$ and $\vec{y}$ with $\|\vec{y}\|_2 < \epsilon$ for which
\begin{equation*}
  (A-\sigma B) \vec{v} = B\vec{y}.
\end{equation*}

If $B^{-1}A$ is normal then it is easily seen that
$\delta(\sigma) = \min_i |\lambda_i - \sigma|$, where the minimum is taken over
all generalized eigenvalues $\lambda_i$ of $(A,B)$.  This also generalizes to
the case in which $B$ is singular.
\begin{lemma}
  \label{lm:normal_eig_bound}
  Assume that $(A,B)$ is a regular pair with an orthonormal basis of
  generalized eigenvectors $\vec{v}_i$, $i = 1, \ldots, n$. Then for
  $\sigma \in \widehat{C}$ we have
  \begin{equation}
    \label{eq:normal_delta}
    \delta(A,B,\sigma) = \min_i |\lambda_i - \sigma|,
  \end{equation}
  where we assume $|\lambda_i - \sigma| = 0$ if $\lambda_i=\sigma=\infty$ and
  $|\lambda_i - \sigma| = \infty$ if exactly one of $\lambda_i$ or $\sigma$
  equals $\infty$.
\end{lemma}
\begin{proof}
  The claim follows immediately with $\delta(A,B,\sigma)=0$ if $\sigma$ is a
  generalized eigenvalue of $(A,B)$, including the case of $\sigma = \infty$.
  If $\sigma = \infty$ is not a generalized eigenvalue, then
  $\delta(A,B,\infty)=\infty$ and $\min_i |\lambda_i -\infty| = \infty$.

  We consider the more interesting case $\sigma\in\C$, where
  $\sigma\notin\Lambda_0(A,B)$.  If $B=0$, then $(A,B)$ has no finite
  generalized eigenvalues and $\delta(A,B,\sigma)=\infty$, which is consistent
  with \eqref{eq:normal_delta}.  So we assume that $B\neq 0$.  Let $V$ be the
  unitary matrix with $i$th column equal to $\vec{v}_i$.  For each generalized
  eigenvector we have $\beta_i A\vec{v}_i = \alpha_i B \vec{v}_i$, where
  regularity ensures that we do not have both $A\vec{v}_i=\vec{0}$ and
  $B\vec{v}_i= \vec{0}$.  It follows that the set $\{A\vec{v}_i, B\vec{v}_i\}$
  spans a subspace of dimension one.  Let $\vec{s}_i\neq \vec{0}$ be in this
  space and let $S$ be the matrix with $i$th column equal $\vec{s}_i$ so that
  \begin{equation*}
    AV = S D_a, \eqand
    BV = S D_b
  \end{equation*}
  for diagonal matrices $D_a$ and $D_b$.  Thus
  \begin{equation*}
    (A-\sigma B) V = S (D_a - \sigma D_b).
  \end{equation*}
  We have assumed that $V$ is unitary and that $\sigma \notin \Lambda_0(A,B)$
  so that $(A-\sigma B)V$ must be nonsingular, which implies that $S$ is
  nonsingular.  Let the diagonal elements of $D_a$ be $\hat{\alpha}_i$ and
  those of $D_b$ be $\hat{\beta}_i$ for $i=1,\ldots, n$.  We have
  \begin{align*}
    A -BE - \sigma B 
    & = S (D_a - \sigma D_b - D_b V^\H E V) V^\H \\
    & = S (D_a - \sigma D_b)\left( I - (D_a - \sigma D_b)^{-1} D_b V^\H E V \right) V^\H.
  \end{align*}
  Let $E \in \mcal{E}(A,B,\sigma)$.  Then
  $I - (D_a - \sigma D_b)^{-1} D_b V^\H E V$ is singular which implies
  \begin{equation*}
    \|(D_a - \sigma D_b)^{-1} D_b\|_2 \|V^\H E V\|_2 \geq \|(D_a - \sigma D_b)^{-1} D_b V^\H E V\|_2 \geq 1.
  \end{equation*}
  We have
  \begin{equation*}
    (D_a - \sigma D_b)^{-1} D_b = \diag\left(\frac{1}{(\lambda_1-\sigma)},
      \ldots, \frac{1}{(\lambda_n-\sigma)}\right)
  \end{equation*}
  so that for any $E\in \mcal{E}(A,B,\sigma)$ we have
  \begin{multline}
    \label{eq:E_lower_bound}
    \|E\|_2 
    \geq\frac{1}{\|(D_a - \sigma D_b)^{-1} D_b\|_2}
    = \frac{1}{\max_i |\hat{\beta}_i|/|\hat{\alpha}_i - \sigma \hat{\beta}_i|}
    = \frac{1}{\max_i 1/|\lambda_i - \sigma|} \\
    = \min_i |\lambda_i - \sigma|.
  \end{multline}
  The pair $(A,B)$ has a finite generalized eigenvalue for each $\hat{\beta}_i$
  satisfying $\hat{\beta}_i\neq 0$.  Since we have assumed that
  $B = S D_b V^\H\neq 0$, we must have at least one nonzero $\hat{\beta}_i$ so
  that $\min_i |\lambda_i - \sigma|$ is finite.  Without loss of generality,
  assume that the columns of $V$ and $S$ are permuted so that
  $|\lambda_1 - \sigma| = \min_i |\lambda_i - \sigma|$ and
  $\hat{\beta}_1\neq 0$.  Defining
  \begin{equation*}
    E_0 = \frac{\hat{\alpha}_1 - \sigma \hat{\beta}_1}{\hat{\beta}_1} V \vec{e}_1 \vec{e}_1^\T V^\H =
    (\lambda_1 - \sigma) V \vec{e}_1 \vec{e}_1^\T V^\H
  \end{equation*}
  ensures that $ I - (D_a - \sigma D_b)^{-1} D_b V^\H E_0 V$ is singular with
  $\vec{e}_1$ as a null vector.  Hence $A-BE_0 - \sigma B$ is singular and
  $E_0\in \mcal{E}(A,B,\sigma)$.  Since $\|E_0\|_2 = |\lambda_1 -\sigma|$,
  $E_0$ achieves the lower bound (\ref{eq:E_lower_bound}), and therefore
  $\delta(A,B,\sigma) = \|E_0\|_2 = |\lambda_1 -\sigma|$.
\end{proof}

\begin{lemma}
  \label{lm:bauer_fike_bound}
  Let $(A,B)$ regular.  Assume that the columns of $V$ give a complete set of
  generalized eigenvectors for $(A,B)$.  Then
  \begin{equation}
    \label{eq:bauer-fike-bound}
    \frac{1}{\delta(A,B,\sigma)}
    \leq \frac{\kappa_2(V)}{\min_i |\lambda_i - \sigma|},
  \end{equation}
  where for $\lambda_i=\infty$ or $\sigma = \infty$ we make the
  same assumptions as in Lemma~\ref{lm:normal_eig_bound}.
\end{lemma}
\begin{proof}
  The case $\sigma = \infty$ is the same as in the proof of
  Lemma~\ref{lm:normal_eig_bound}, with $\delta(A,B,\sigma) = 0$ or
  $\delta(A,B,\sigma)=\infty$, depending on whether $(A,B)$ has an infinite
  eigenvalue.  For $\sigma\in\C$, the case $B=0$ and $\sigma\in \Lambda_0(A,B)$
  are also similar.  So we assume that $\sigma\in\C$, $B\neq 0$, and $A-\sigma B$
  is nonsingular.  The rest of the proof is also similar to the first part of
  Lemma~\ref{lm:bauer_fike_bound}.  Let
  $\beta_i A \vec{v}_i = \alpha_i B \vec{v_i}$ with $\alpha_i$ and $\beta_i$
  not both zero.  We can construct a nonsingular $S$ such that $A V = SD_a$ and
  $BV= SD_b$, where the diagonal elements of $D_a$ are $\hat{\alpha}_i$ and
  those of $D_b$ are $\hat{\beta}_i$.  We then have
  \begin{multline*}
    \|(A-\sigma B)^{-1} B\|_2
    = \|V (D_a - \sigma D_b)^{-1} D_b V^{-1}\|_2
    \leq \kappa_2(V) \|(D_a - \sigma D_b)^{-1} D_b \|_2 \\
    = \frac{\kappa_2(V)}{\min_i | \lambda_i - \sigma|}.
  \end{multline*}
\end{proof}

In addition to the pseudospectral separation $\delta(A,B,\sigma)$, for
$\sigma \in \C$, $\sigma\neq 0$, and $\sigma\notin\Lambda_0(A,B)$ we are
interested in the \textit{relative pseudospectral separation}
$\delta(A,B,\sigma)/|\sigma|$.  With normalized $\hat{A}$, $\hat{B}$, and
$\hat{\sigma}$ as in \eqref{eq:scaled_eig_shift}, we have
\begin{equation*}
  \frac{\delta(A,B,\sigma)}{|\sigma|}
  = \frac{1}{|\hat{\sigma}|\|A\|_2 \|(A-\sigma B)^{-1}B\|_2/\|B\|_2}
  = \frac{1}{|\hat{\sigma}| \|(\hat{A}-\hat{\sigma} \hat{B})^{-1}\hat{B}\|_2}
  = \frac{\delta(\hat{A},\hat{B},\hat{\sigma})}{|\hat{\sigma}|}.
\end{equation*}
Hence, assuming that $\sigma$ is scaled in a manner consistent with scaling of
$A$ and $B$, the relative pseudospectral separation does not depend on the
scaling of $A$ and $B$.

\section{Pseudospectral Bounds on $K_2(A,B)$}
\label{sec:pseud-bounds-k2}

In this section, we assume that $\sigma \neq \infty$ and
$\sigma \notin \Lambda_0(A,B)$ and consider the magnitude of the quantity
$K_2(A-\sigma B, B)$ introduced in section~\ref{sec:background}.  We begin by
proving the bounds \eqref{eq:upper_lower_bounds}.
\begin{lemma}
  Let $\sigma \neq \infty$ and $\sigma\notin \Lambda_0(A,B)$.  Then the
  quantity $K_2(A-\sigma B, B)$ satisfies \eqref{eq:upper_lower_bounds}.
\end{lemma}
\begin{proof}
  The upper bound is immediate from
  \begin{multline*}
    K_2(A-\sigma B, B)
    = \frac{\|A-\sigma B\|_2}{\|B\|_2} \|(A-\sigma B)^{-1} B\|_2
    \leq \frac{\|A-\sigma B\|_2}{\|B\|_2} \|(A-\sigma B)^{-1}\|_2 \|B\|_2 \\
    = \|A-\sigma B\|_2 \|(A-\sigma B)^{-1}\|_2
    = \|\hat{A}-\hat{\sigma} \hat{B}\|_2 \|(\hat{A}-\hat{\sigma} \hat{B})^{-1}\|_2.
  \end{multline*}
  For the lower bound, we let $A = Q T_a Z^\H$ and $B = Q T_b Z^\H$ be a
  generalized Schur decomposition of $A$ and $B$, where the diagonal elements
  of $T_a$ are $\alpha_i$ and the diagonal elements of $T_b$ are $\beta_i$.  Then
  \begin{equation*}
    \|(A-\sigma B)^{-1} B\|_2 = \|T\|_2,\qquad\text{where}\qquad
    T = (T_a - \sigma T_b)^{-1} T_b.
  \end{equation*}
  We then have
  \begin{equation*}
    \|(A-\sigma B)^{-1} B\|_2 
    \geq \rho(T) 
    = \max_i \left| \frac{\beta_i}{\alpha_i - \sigma \beta_i} \right|
    = \max_i \frac{1}{|\lambda_i - \sigma|}
    = \frac{1}{\min_i |\lambda_i - \sigma|},
  \end{equation*}
  where $\rho(T)$ is the spectral radius of $T$.  It should be understood that
  if $\beta_i=0$, then $\lambda_i = \infty$ and $1/|\lambda_i -\sigma|=0$.  With scaling, we have
  \begin{equation*}
    K_2(A-\sigma B, B)
    \geq \frac{\|A-\sigma B\|_2}{\|A\|_2} 
     \frac{1}{\min_i \|B\|_2 |\lambda_i - \sigma|/\|A\|_2}
     = \|\hat{A} - \hat{\sigma} \hat{B}\|_2
     \frac{1}{\min_i |\hat{\lambda}_i - \hat{\sigma}|}.
  \end{equation*}
\end{proof}

To interpret the lower bound, we define
\begin{equation}
  \label{eq:gamma_def}
  \gamma = \frac{\|A-\sigma B\|_2}{\|A\|_2} 
  = \|\hat{A}-\hat{\sigma} \hat{B}\|_2 \leq 1+|\hat{\sigma}|.
\end{equation}
This quantity can be large only if the scaled shift $\hat{\sigma}$ is large.
We might hope to moderate $K_2(A-\sigma B, B)$ by choosing $\sigma$ so that
$\hat{\sigma}$ is not large.  However we have
\begin{equation}
  \label{eq:upper_bound_for_lower_bound}
  \frac{\|\hat{A}-\hat{\sigma} \hat{B}\|_2}{\min_i |\hat{\lambda_i} - \hat{\sigma}|}
  \leq \frac{1+|\hat{\sigma}|}{\min_i |\hat{\lambda_i} - \hat{\sigma}|}
  = \frac{1+ 1/|\hat{\sigma}|}{\min_i |1 - \hat{\lambda_i}/\hat{\sigma}|}
  = \frac{1+ 1/|\hat{\sigma}|}{\min_i |1 - \lambda_i/\sigma|}.
\end{equation}
so that for larger scaled shifts, the magnitude of the lower bound is
constrained by the relative distance of $\sigma$ from a generalized eigenvalue.
For small scaled shifts, the absolute distance is more important.

From Lemma~\ref{lm:resolvent_bound} we know that
\begin{equation*}
  K_2(A-\sigma B, B) = \frac{\|A-\sigma B\|_2}{\|B\|_2} \frac{1}{\delta(\sigma)}.
\end{equation*}
We would prefer to have a version of this identity that
is independent of the scaling of $A$ and $B$.  Hence
\begin{align}
  K_2(A-\sigma B, B) 
  & = \frac{\|A-\sigma B\|_2}{\|B\|_2} \|(A-\sigma B)^{-1}B\|_2 \nonumber\\
  & = \left( \frac{\|A-\sigma B\|_2}{\|B\|_2} \frac{\|B\|_2}{\|A\|_2}\right) 
    \left( \frac{\|A\|_2}{\|B\|_2}\|(A-\sigma B)^{-1}B\|_2  \right)\nonumber\\
  & = \left\| \frac{A}{\|A\|_2} - \sigma \frac{\|B\|_2}{\|A\|_2}
    \frac{B}{\|B\|_2}
    \right\|_2 \left\|\left(\frac{A}{\|A\|_2}
    - \sigma \frac{\|B\|_2}{\|A\|_2} \frac{B}{\|B\|_2}\right)^{-1}
    \frac{B}{\|B\|_2}\right\|_2 \nonumber \\
  & = \|\hat{A} - \hat{\sigma} \hat{B}\|_2 
    \|(\hat{A} - \hat{\sigma} \hat{B})^{-1} \hat{B}\|_2 \nonumber\\
  & = \frac{\gamma}{\delta(\hat{A}, \hat{B}, \hat{\sigma})},\label{eq:K2-delta-ident}
\end{align}
where, as in section~\ref{sec:background}, $\hat{A} = A/\|A\|_2$,
$\hat{B} = B/\|B\|_2$, and $\hat{\sigma}$ is the scaled shift in
(\ref{eq:scaled_eig_shift}).  Note that \eqref{eq:K2-delta-ident} is an
identity and not a bound.  The identity suggests two equivalent ways of writing
an upper bound for $K_2(A-\sigma B, B)$.  Using the fact that
$\gamma \leq 1 + |\hat{\sigma}|$ gives
\begin{equation*}
  K_2(A-\sigma B, B)
  \leq \frac{1+|\hat{\sigma}|}{\delta(\hat{A}, \hat{B}, \hat{\sigma})}
  = \frac{1+1/|\hat{\sigma}|}{\delta(\hat{A}, \hat{B}, \hat{\sigma})/|\hat{\sigma}|}
  = \frac{1+1/|\hat{\sigma}|}{\delta(A, B, \sigma)/|\sigma|}.
\end{equation*}
Thus $K_2(A-\sigma B)$ can be expected to be of moderate magnitude in two
cases: if $|\hat{\sigma}|$ is not large and the pseudospectral distance of
$\hat{\sigma}$ from $\Lambda_0(\hat{A}, \hat{B})$ is not small or if
$|\hat{\sigma}|$ is not small and the pseudospectral distance of $\hat{\sigma}$
from $\Lambda_0(\hat{A}, \hat{B})$ is not small relative to the magnitude of
$|\hat{\sigma}|$.

Finally, we can bound $K_2(A-\sigma B, B)$ in terms of the condition number of
the matrix of generalized eigenvectors.
\begin{lemma}
  \label{lm:bauer_fike_bound_K2}
  Let $B\neq 0$ and $\sigma \notin \Lambda_0$.  Assume that the columns of $V$ give a
  complete set of eigenvectors for $(A,B)$.  Then
  \begin{equation*}
    K_2(A-\sigma B, B) 
    \leq \kappa_2(V) \frac{\|\hat{A} - \hat{\sigma} \hat{B}\|_2}
    {\min_i |\hat{\lambda}_i - \hat{\sigma}|}
    \leq \kappa_2(V) \frac{1+|\hat{\sigma}|}{\min_i |\hat{\lambda}_i - \hat{\sigma}|}
    = \kappa_2(V) \frac{1+1/|\hat{\sigma}|}{\min_i |1 - \hat{\lambda}_i / \hat{\sigma}|}.
  \end{equation*}
\end{lemma}
\begin{proof}
  The proof uses the same approach as the start of the proof of
  Lemma~\ref{lm:normal_eig_bound}.  Let
  $\beta_i A \vec{v}_i = \alpha_i B \vec{v_i}$ with $\alpha_i$ and $\beta_i$ no
  both zero.  As in Lemma~\ref{lm:normal_eig_bound} we can construct an $S$
  such that $A V = SD_a$ and $BV= SD_b$ where the diagonal elements of $D_a$
  are $\hat{\alpha}_i$ and those of $D_b$ are $\hat{\beta}_i$.  We then have
  \begin{multline*}
    \|(A-\sigma B)^{-1} B\|_2
    = \|V (D_a - \sigma D_b)^{-1} D_b V^{-1}\|_2
    \leq \kappa_2(V) \|(D_a - \sigma D_b)^{-1} D_b \|_2 \\
    = \kappa_2(V) \frac{1}{\min_i | \lambda_i - \sigma|}.
  \end{multline*}
  The bounds involving scaled matrices and quantities are given by the same
  manipulation we have used in \eqref{eq:upper_bound_for_lower_bound}.
\end{proof}

\section{Residual Bounds}
\label{sec:residual-bounds}

In this section, we prove residual bounds for the shift and invert solution of
the generalized eigenvalue problem using \eqref{eq:X-def} and
\eqref{eq:X_problem}.  We begin by making assumptions about the errors in each
step of the algorithm.  Unless otherwise noted, matrices and vector correspond
to computed quantities.  Defining $\breve{A} = \fl(A-\sigma B)$, we assume that
\begin{equation}
  \label{eq:shift_error}
  \|\breve{A} - (A-\sigma B)\|_2 \leq u (a_n \|A-\sigma B\|_2 + b_n |\sigma| \|B\|_2)
\end{equation}
where $u$ is the unit roundoff and $a_n$ and $b_n$ depend on $n$ only and not
on $A$, $B$, or $\sigma$.  This bound is what would be expected from the
computation $\breve{a}_{ij} = \fl(a_{ij} - \sigma b_{ij})$.  If a fused
multiply-add is used, then we can let $b_n=0$, which simplifies some of the
error bounds.  For the computation of $X$, we assume that
\begin{equation}
  \label{eq:X-error}
  \left\| \breve{A} X - B \right\|_2 
  \leq u c_n \|\breve{A}\|_2\|X\|_2,
\end{equation}
with similar assumptions on $c_n$.  A bound of this form holds if
$\breve{A}X = B$ is solved using a backward stable algorithm.  With these
assumptions, the following theorem connects residuals for the ordinary
eigenvalue problem \eqref{eq:X_problem} to residuals for \eqref{eq:gen_eig2}.
\begin{theorem}
  \label{th:X-residuals}
  Suppose that \eqref{eq:shift_error} and \eqref{eq:X-error} hold.  Let
  $\vec{v}\neq \vec{0}$ and $\theta \neq 0$ satisfy
  \begin{equation}
    \label{eq:X-residual}
    X\vec{v} = \theta \vec{v} + \vec{r}.
  \end{equation}
  Let $\lambda = \sigma + 1/\theta$ with scaled quantities
  $\hat{\lambda}$ and $\hat{\sigma}$ as in \eqref{eq:scaled_eig_shift} and
  $\gamma$ as in \eqref{eq:gamma_def}.  If $\lambda \neq 0$, then
  \begin{multline}
    \label{eq:residual-b}
    \| \theta A \vec{v} - (1+\sigma \theta) B \vec{v}\|_2
    \leq |1+\sigma \theta| \cdot | 1- \hat{\sigma} / \hat{\lambda}|
    \Bigg(u \left(a_n+ 2b_n\min(1/\gamma, 1) + c_n\right)  \\
    + \frac{\|\vec{r}\|_2}{\|X\|_2 \|\vec{v}\|_2}\Bigg)
    K_2(A-\sigma B, B) \|B\|_2 \|\vec{v}\|_2.
  \end{multline}
  If $\lambda \neq \infty$, we have
  \begin{multline}
    \label{eq:residual-a}
    \| \theta A \vec{v} - (1+\sigma \theta) B \vec{v}\|_2
    \leq |\theta| \cdot | \hat{\lambda} - \hat{\sigma}|
    \Bigg(u \left(a_n+ 2b_n \min(1/\gamma, 1) + c_n\right) \\
    + \frac{\|\vec{r}\|_2}{\|X\|_2 \|\vec{v}\|_2}\Bigg)
    K_2(A-\sigma B,B) \|A\|_2 \|\vec{v}\|_2.
  \end{multline}
\end{theorem}
\begin{proof}
  Let
  \begin{equation*}
    \breve{A} = A - \sigma B + G_0, \eqand
    \breve{A} X - B = G_1,
  \end{equation*}
  so that multiplying \eqref{eq:X-residual} by $\breve{A}$ gives
  \begin{equation*}
    (B+ G_1)\vec{v} = \theta (A-\sigma B + G_0)\vec{v} + \breve{A} \vec{r}
  \end{equation*}
  or
  \begin{equation*}
    \theta A \vec{v} - (1+\sigma \theta) B\vec{v} 
    = G_1 \vec{v} - \theta G_0 \vec{v} - \breve{A} \vec{r}.
  \end{equation*}
  Using \eqref{eq:shift_error} and \eqref{eq:X-error} after taking norms gives
  \begin{alignat}{3}
    \|\theta A \vec{v} - (1+\sigma \theta) B\vec{v}\|_2 
    & \leq && u \Big( a_n |\theta| \|A-\sigma B\|_2 + b_n |\sigma \theta| \|B\|_2 
      + c_n \|\breve{A}\|_2 \|X\|_2 \Big)\|\vec{v}\|_2 \nonumber \\
    & && + \|\breve{A}\|_2 \|X\|_2 \frac{\|r\|_2}{\|X\|_2
         \|\vec{v}\|_2} \|\vec{v}\|_2  \nonumber \\
    & \leq && u \Big( a_n \|A-\sigma B\|_2 \|X\|_2 + b_n |\sigma| \|B\|_2 \|X\|_2 
              + c_n \|\breve{A}\|_2 \|X\|_2 \Big) \nonumber \\
    & &&\quad \cdot \|\vec{v}\|_2 
    + \|\breve{A}\|_2 \|X\|_2 \frac{\|r\|_2}{\|X\|_2 \|\vec{v}\|_2} \|\vec{v}\|_2
         +O(u^2)\nonumber \\
    & = && \Bigg( u \Big( a_n + b_n |\hat{\sigma}|/\gamma
              + c_n \Big) + \frac{\|r\|_2}{\|X\|_2 \|\vec{v}\|_2}\Bigg) \nonumber \\
    & && \cdot K_2(A-\sigma B, B) \|B\|_2 \|\vec{v}\|_2 +O(u^2), \label{eq:X-res-no-eigs}
  \end{alignat}
  where $\hat{\sigma}$ is as in \eqref{eq:scaled_eig_shift} and $\gamma$ is as
  in \eqref{eq:gamma_def}.  We note that
  \begin{equation*}
    \frac{|\hat{\sigma}|}{\gamma}
    = \frac{\hat{\sigma}}{\|\hat{A} - \hat{\sigma} \hat{B}\|_2}
    \leq \frac{|\hat{\sigma}|}{| \|\hat{A}\|_2 - |\hat{\sigma}| \|\hat{B}\|_2|}
    = \frac{|\hat{\sigma}|}{| 1 - |\hat{\sigma}| |}.
  \end{equation*}
  If $|\hat{\sigma}| \leq 2$, then $|\hat{\sigma}|/\gamma \leq 2/\gamma$.  If
  $|\hat{\sigma}| > 2$, then the above inequality implies that
  $|\hat{\sigma}| / \gamma \leq 2$.  Thus
  \begin{equation}
    \label{eq:sigma0-gamma-bound}
    \frac{|\hat{\sigma}|}{\gamma} \leq 2\max\left(\frac{1}{\gamma}, 1\right).
  \end{equation}
  When $\lambda \neq 0$, we have
  \begin{equation*}
    |1+\sigma \theta|
    = \left| 1 +  \frac{\sigma}{\lambda - \sigma}  \right|
    = \frac{1}{|1-\sigma/\lambda|}
    = \frac{1}{|1-\hat{\sigma}/\hat{\lambda}|}
  \end{equation*}
  so that \eqref{eq:residual-b} follows immediately from
  \eqref{eq:X-res-no-eigs} and \eqref{eq:sigma0-gamma-bound}.  When
  $\lambda \neq \infty$ we have
  \begin{equation*}
    \theta 
    = \frac{1}{\lambda-\sigma} 
    = \frac{\|B\|_2}{\|A\|_2 (\hat{\lambda}-\hat{\sigma})}
  \end{equation*}
  so that
  \begin{equation*}
    \|B\|_2 = \theta (\hat{\lambda} - \hat{\sigma}) \|A\|_2
  \end{equation*} 
  and \eqref{eq:residual-a} also follows from \eqref{eq:X-res-no-eigs} and
  \eqref{eq:sigma0-gamma-bound}.
\end{proof}

The bounds in the theorem merit some explanation.  The following is a variation
on theorems from \cite{frto:98, hihi:98} that was used in a similar form in
\cite{stewm:24}.
\begin{lemma}
\label{lm:general-residual-bounds}
Suppose that for some complex $\alpha$ and $\beta$, not both zero, and
$\vec{v}\neq \vec{0}$ we have
\begin{equation*}
  \beta A\vec{v} - \alpha B \vec{v} = \vec{r}.
\end{equation*}
If
\begin{equation}
  \label{eq:relative-residual}
  \|\vec{r}\|_2 \leq (|\beta| \|A\|_2 + |\alpha| \|B\|_2) \|\vec{v}\|_2 \epsilon
\end{equation}
for $\epsilon > 0$, then there exist $E$ and $F$ satisfying
\begin{equation*}
  \max\left( \frac{\|E\|_2}{\|A\|_2}, \frac{\|F\|_2}{\|B\|_2} \right) 
  \leq \epsilon
\end{equation*}
such that
\begin{equation*}
  \beta (A+E)\vec{v} - \alpha (B+F) \vec{v} = \vec{0}.
\end{equation*}
\end{lemma}

Combining the lemma with Theorem~\ref{th:X-residuals} immediately gives the
following result.
\begin{corollary}
  \label{co:backward-errors}
  Under the assumptions of Theorem~\ref{th:X-residuals}, there exist $E$ and $F$
  satisfying
  \begin{equation*}
     \theta (A+E) \vec{v} = (1+\sigma\theta) (B+F) \vec{v}
  \end{equation*}
  with
\begin{multline*}
  \max\left( \frac{\|E\|_2}{\|A\|_2}, \frac{\|F\|_2}{\|B\|_2} \right) 
  \leq \min(| 1- \hat{\sigma} / \hat{\lambda}|, |\hat{\lambda} - \hat{\sigma}|)
  \Bigg(u \left(a_n+ 2b_n \min(1/\gamma, 1) + c_n\right) \\
    + \frac{\|\vec{r}\|_2}{\|X\|_2 \|\vec{v}\|_2}\Bigg)
    K_2(A-\sigma B,B),
\end{multline*}
where we let $| 1- \hat{\sigma} / \hat{\lambda}|=\infty$ if $\lambda = 0$ and
$|\hat{\lambda} - \hat{\sigma}|=\infty$ if $\lambda = \infty$.
\end{corollary}

If a backward stable algorithm is used to solve the ordinary eigenvalue problem
$X\vec{v} = \theta \vec{v}$, then we expect the relative residual
$\|\vec{r}\|_2 /(\|X\|_2 \|\vec{v}\|_2)$ to be not much larger than the unit
roundoff $u$.  The corollary suggests circumstances in which a stable solution
of the ordinary eigenvalue problem might fail to give an eigenvalue
$(\theta, 1+\sigma \theta)$ and eigenvector $\vec{v}$ that will fail to
correspond to a nearby pair $(A+E, B+F)$.  The most significant point of
failure is if $K_2(A-\sigma B, B)$ is large.  From
section~\ref{sec:pseudospectra} section~\ref{sec:pseud-bounds-k2}, we know that
to avoid this, we need to choose $\sigma$ to be well separated from
$\Lambda_0(A,B)$ in a pseudospectral sense.  This tends to be relatively easy
to do if the pair $(A,B)$ has a well conditioned set of generalized
eigenvectors, regardless of the condition number of $A-\sigma B$.

There are several other factors that can influence the magnitude of the bounds.
We note that $1/\gamma = \|A\|_2 / \|A-\sigma B\|_2$ is large only if there is
extreme cancellation in forming $A-\sigma B$.  This seems unlikely in practice.

Finally, we consider the quantity
$\min(|1-\hat{\sigma}/\hat{\lambda}|, |\hat{\lambda} - \hat{\sigma}|)$.  There
are several cases of interest.
\begin{enumerate}
\item The scaled shift $\hat{\sigma}$ is not large: In
  this case, if $|\hat{\lambda}| \geq 1$ then
  $|1-\hat{\lambda}/\hat{\sigma}| \leq 1 + |\hat{\sigma}|$.  If
  $|\hat{\lambda}| \leq 1$, then
  $|\hat{\lambda} - \hat{\sigma}| \leq 1 + |\hat{\sigma}|$.  So, either way, we
  have
  \begin{equation*}
    \min(|1-\hat{\sigma}/\hat{\lambda}|, |\hat{\lambda} - \hat{\sigma}|)
    \leq 1 + |\hat{\sigma}|.
  \end{equation*}
  If the scaled shift is not large, then this factor does not increase the
  magnitude of the residual and backward error bounds.
\item The scaled shift $\hat{\sigma}$ is large: In this case,
  $|1-\hat{\sigma}/\hat{\lambda}|$ will not be large when $\hat{\lambda}$ is
  not much smaller in magnitude than $\hat{\sigma}$.  So, for a large scaled
  shift, this factor in the error bounds is harmless for eigenvalues that are
  also large, but might cause problems with small eigenvalues.
\end{enumerate}

\section{Schur Decompsition}
\label{sec:schur-decomp}

We now consider the computation of a generalized Schur decomposition of $(A,B)$
from a schur decomposition of $X = (A-\sigma B)^{-1} B$.  Suppose that
$\sigma \notin \Lambda_0(A,B)$ and that we have a Schur decomposition
$X = ZTZ^\H$.  We then have
\begin{equation*}
  (A -\sigma B) Z T = B Z.
\end{equation*}
If we compute a $QR$ factorization $(A -\sigma B)Z = QT_1$, then $T_1$ is
nonsingular and $Q^\H (A-\sigma B) Z = T_1$ so that
\begin{equation*}
  Q^\H BZ = Q^\H (A-\sigma B)Z T = T_1 T,\eqand
  Q^\H A Z = T_1 + \sigma Q^\H BZ = T_1 (I+\sigma T).
\end{equation*}
This leads to a generalized Schur decomposition
\begin{equation}
  \label{eq:Gen_schur_decomp}
  Q^\H A Z = T_1(I+\sigma T),\eqand Q^\H BZ = T_1 T.
\end{equation}
Clearly $T_1$ is nonsingular and the generalized eigenvalues and eigenvectors
of $(A,B)$ are the same as those of $(I+\sigma T, T)$.  If the diagonal
elements of $T$ are $\theta_i$ for $i=1,\ldots n$, then in \eqref{eq:gen_eig2}
the eigenvalues of $(A,B)$ are $(1+\sigma\theta_i, \theta_i)\neq (0,0)$.  The
first $k$ columns of $Z$ give an orthonormal basis for a deflating subspace of
$(A,B)$.  The computation is summarized in Algorithm~\ref{alg:gen_schur}.

The full generalized Schur decomposition is more than is needed for generalized
eigenvalues and eigenvectors, which can be computed directly from the Schur
decomposition $X = ZTZ^\T$ in the standard way.  In this case the residual
bounds of Theorem~\ref{th:X-residuals} apply with a suitably small residual
$\vec{r}$.  While Algorithm~\ref{alg:gen_schur} has no particular benefit in
efficiency or stability for simply computing eigenvalues, there are other
problems in which a full generalized Schur decomposition is useful and we will
consider the stability properties of this decomposition.

\begin{algorithm}
\caption{Shift and Invert Schur Decomposition}
\label{alg:gen_schur}
\begin{algorithmic}
\Function{GenSchur}{$A, B, \sigma$}
\State $\breve{A} \gets A - \sigma B$
\State Solve: $\breve{A}X = B$
\State Factor: $X = Z T Z^{\H}$
\State $W \gets \breve{A}Z$
\State Factor: $W = Q T_1$
\State \Return $(T, T_1, Q, Z)$
\EndFunction
\end{algorithmic}
\end{algorithm}

\begin{theorem}
  \label{th:generalized_schur}
  Suppose that $\sigma \notin \Lambda_0(A,B)$ and that \eqref{eq:shift_error}
  and \eqref{eq:X-error} hold.  Also assume that
  \begin{equation*}
    % \label{eq:schur_error}
    \| X - \tilde{Z} T \tilde{Z}^\H \|_2 \leq u d_n \|X\|_2,\qquad
    \|Z - \tilde{Z}\|_2 \leq u e_n,
  \end{equation*}
  \begin{equation}
    \label{eq:AZerror}
    \|\breve{A} Z - W\|_2 \leq u f_n \|\breve{A}\|_2,
  \end{equation}
  \begin{equation}
    \label{eq:QRerror}
    \|\tilde{Q}T_1 - W\|_2 \leq u g_n \|W\|_2, \eqand
    \|\tilde{Q} - Q\|_2 \leq uh_n ,
  \end{equation}
  where $\tilde{Q}$ and $\tilde{Z}$ are unitary, $T_1$ and $T$ are
  upper triangular, and $T$ is nonsingular.  Then there exist $E$ and
  $F$ satisfying
  \begin{equation*}
    \tilde{Q}^\H (A+E) \tilde{Z} = T_1(I+\sigma T),\eqand \tilde{Q}^\H (B+F) \tilde{Z}
    = T_1 T.
  \end{equation*}
  with
  \begin{alignat*}{3}
    \frac{\|E\|_2}{\|A\|_2} & \leq
    u \Big(&& |\hat{\sigma}| (c_n + d_n + e_n + f_n + g_n) K_2(A-\sigma B, B) \\
    & && + (1+|\hat{\sigma}|)(a_n + b_n + e_n + f_n + g_n)\Big) +O(u^2)    
  \end{alignat*}
  and
  \begin{equation*}
    \frac{\|F\|_2}{\|B\|_2} \leq
    u \left( c_n +d_n +e_n + f_n + g_n\right) K_2(A-\sigma B, B) + O(u^2).
  \end{equation*}
\end{theorem}
\begin{proof}
  Let $\breve{A} X - B = G_1$, $X - \tilde{Z}T \tilde{Z}^\H = G_2$,
  $Z - \tilde{Z} = G_3$, $\breve{A}Z - W = G_4$, and
  $\tilde{Q}T_1 - W = G_5$.  The first two of these identities give
  \begin{equation*}
    \breve{A} \tilde{Z} T = B\tilde{Z} + G_1 \tilde{Z} - \breve{A}G_2 \tilde{Z}.
  \end{equation*}
  Continuing with various substitutions, we have
  \begin{equation*}
    \breve{A} Z T = B\tilde{Z} + G_1 \tilde{Z} - \breve{A}G_2 \tilde{Z} + \breve{A} G_3 T,
  \end{equation*}
  \begin{equation*}
    W T = B\tilde{Z} + G_1 \tilde{Z} - \breve{A}G_2 \tilde{Z} + \breve{A} G_3 T - G_4 T,
  \end{equation*}
  and
  \begin{equation*}
    \tilde{Q} T_1 T = B\tilde{Z} + G_1 \tilde{Z} - \breve{A}G_2 \tilde{Z} + \breve{A} G_3 T - G_4 T + G_5 T.
  \end{equation*}
  This implies that
  \begin{equation*}
    B + F = \tilde{Q} T_1 T \tilde{Z}^\H
  \end{equation*}
  where
  \begin{alignat*}{3}
    \|F\|_2
    & \leq && \|G_1\|_2 + \|\breve{A}\|_2 \|G_2\|_2 + \|\breve{A}\|_2 \|G_3\|_2 \|T\|_2 +
              \|G_4\|_2 \|T\|_2 + \|G_5\|_2 \|T\|_2 \\
    & = && \|G_1\|_2 + \|\breve{A}\|_2 \|G_2\|_2 + \|\breve{A}\|_2 \|G_3\|_2 \|X\|_2 +
           \|G_4\|_2 \|X\|_2 + \|G_5\|_2 \|X\|_2   \\
    & && + O(u^2) \\
    & \leq && u ( c_n \|\breve{A}\|_2 \|X\|_2 + d_n \|\breve{A}\|_2 \|X\|_2 
              + e_n \|\breve{A}\|_2 \|X\|_2\\
    & && + f_n \|\breve{A}\|_2 \|X\|_2 + g_n \|W\|_2 \|X\|_2) +O(u^2) \\
    & = && u \left( c_n +d_n +e_n + f_n + g_n\right) \|A-\sigma B\|_2 \|X\|_2 + O(u^2) \\
    & = && u \left( c_n +d_n +e_n + f_n + g_n \right) K_2(A-\sigma B, B) \|B\|_2 + O(u^2).
  \end{alignat*}
  
  Let $\breve{A} = A-\sigma B + G_0$.  Considering errors on $A$, we
  start with $W - G_5 = \tilde{Q}T_1$ and continue with substitutions
  \begin{equation*}
    \breve{A}Z - G_4 - G_5 = \tilde{Q}T_1,
  \end{equation*}
  \begin{equation*}
    (A-\sigma B + G_0)Z - G_4 - G_5 = \tilde{Q}T_1,
  \end{equation*}
  and
  \begin{equation*}
    (A-\sigma B + G_0)\tilde{Z} - G_4 + (A-\sigma B+G_0) G_3 - G_5  = \tilde{Q}T_1.
  \end{equation*}
  This implies that
  \begin{multline*}
    \tilde{Q}^\H \left(A + \sigma F + G_0 + (A-\sigma B+G_0) G_3\tilde{Z}^\H
      - G_4\tilde{Z}^\H -G_5\tilde{Z}^\H\right)
    \tilde{Z} \\
    = T_1 + \sigma \tilde{Q}^\H(B+F) \tilde{Z} = T_1 (I+\sigma T),
  \end{multline*}
  or
  \begin{equation*}
    \tilde{Q}^\H (A+E) \tilde{Z} = T_1(I+\sigma T)
  \end{equation*}
  where
  \begin{equation*}
    E = \sigma F + G_0 + (A-\sigma B+G_0) G_3\tilde{Z}^\H
    - G_4\tilde{Z}^\H -G_5\tilde{Z}^\H
  \end{equation*}
  and
  \begin{alignat*}{3}
    \|E\|_2 
    & \leq && |\sigma| \|F\|_2 + \|G_0\|_2 + \|A-\sigma B\|_2 \|G_3\|_2 + \|G_4\|_2 
              + \|G_5\|_2 + O(u^2) \\
    & \leq &&  u |\sigma| (c_n + d_n + e_n + f_n + g_n) K_2(A-\sigma B, B) \|B\|_2 
              + ub_n |\sigma| \|B\|_2 \\
    & && + u(a_n + e_n + f_n + g_n) \|A-\sigma B\|_2 +O(u^2)\\
    & \leq && u |\hat{\sigma}| (c_n + d_n + e_n + f_n + g_n) 
          K_2(A-\sigma B, B) \|A\|_2 + ub_n |\hat{\sigma}| \|A\|_2 \\
    & && + u(a_n + e_n + f_n + g_n) (\|A\|_2 + |\hat{\sigma}| \|A\|_2) +O(u^2)\\
    & \leq && u \Big( |\hat{\sigma}| (c_n + d_n + e_n + f_n + g_n) K_2(A-\sigma B, B) \\
    & && \quad + (1+|\hat{\sigma}|)(a_n + b_n + e_n+f_n + g_n)\Big) \|A\|_2 +O(u^2).\\
  \end{alignat*}
\end{proof}




\bibliography{/home/mas/work/bib/ref}


\end{document}
